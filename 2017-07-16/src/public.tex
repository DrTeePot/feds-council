\section*{Consent}

\heading{Approval of the minutes}
\begin{motion}
    \birt Board approve the minutes from May 28, 2017.
    \movers{\tristan}{\seneca}

    There was one minor correction, which will be emailed and corrected in
    the final version.

    \carries unanimously.
\end{motion}

\heading{Councillor Reports}
\begin{information}
    The Environment Councillors made a report, touching on some renovations
    that were occuring and some changes they were pushing within the 
    university.
\end{information}

\section*{Executive Reports}

Action plans are still being developed, and will be presented at the 
next meeting. 

\heading{Report of the \pres}

The \pres\ was not present, but submitted a written report (attached). 

\heading{Report of the \vpi}

\begin{information}
    
    The \vpi\ reported that a Feds Wellness Committee had been created, 
    comprised of 3 people internal to Feds. They are continuing to connect with
    stakeholders on campus and being careful not to overstep. The aim is to 
    ensure Feds is doing the best they can to support students, including 
    reaching out to various university services to have more partnerships. 
    The committee is also doing a critical analysis of wrap-up week and 
    mental health week. 

    They also reported that one service, the Women's Center, had reached out 
    about centralizing advocacy efforts. The \vpi\ is hoping to have the other 
    5 services become part of this pilot inniative as well. 

    Feds on Tour did not happen; instead, the \vpi\ is investigating how 
    to make it more effective for all students, and is reaching out to 
    satellite campus' and societies to improve outreach. 

\end{information}

\heading{Report of the \vpof}
\begin{information}

    The \vpof\ highlighted that special bomber events were a success, 
    with over 200 people attending on average. As well, Beerfest was similarly
    successful, with students providing the feedback that they prefered that 
    it was an relaxed evening event instead of a day or night event. 

    The \vpof\ continues to push to make the Bomber more sustainable,
    improving environmental practises within the organization. 

    Feds also provided wrap-up week deals for all commercial services,
    which was well recieved by students. 

    Wasabi and campus bubble are re-doing their menus, students are invited
    to taste test the new ideas being introduced, with samples being provided 
    in the great hall. 

    In external news, the \vpof\ is continuing to monitor the ongoing 
    discussions around Pharmacare plus. The government has not released any new
    updates. 

    The budget committee was officially formed at the last Board meeting, 
    and the budget has been compiled. It is in the process of being finalized, 
    and will be presented at the next meeting. 

    A councillor inquired what was meant by action planning, specifically 
    whether that was long or short term planning. The \vpof\ responded that 
    they were annual plans to accomplish their platform points, to provide 
    strategy for accomplishing goals, and to provide students with transparency
    to hold them accountable.  

    Finally, the \vpof\ reported that \seneca\ was elected as councillor 
    representative, Mathew Gerrits as board representative, and Graham Barnes 
    as at-large representative to the budget committee. 

\end{information}

\heading{Report of the \vpe}
\begin{information}

    The \vpe\ began by apologizing for the lack of written report. 
    
    They unvieled a land-acknowledgement plaque, which is the first public 
    action in response to the truth and reconciliation commission. There will 
    be a report that goes to the Provost's Advisory Committee on Equity about 
    further actions. 

    The \vpe\ also reported that the Municipal Affairs Commissioner is 
    starting a project on what transit and safety should look like on campus, 
    and how that will integrate with the transit solution in the rest of the 
    region. 

    The Acadmic support has gotten a lot better, the commission is improving their
    back end processes to increase capacity so that the commission can help 
    more students with their grievances.

    Additionally, Feds has met twice with the university committee behind the 
    coop fee review.  They are creating a student advisory board that will be 
    appointed for a year, made of at-large students. More information 
    will be presented at a later date. 

    OUSA priorities have been set, they will be: sexual violence prevention and 
    response, data collection, open education (open materials, reduce costs), 
    mental health funding, experiential education (including coop), and 
    a funding review. 

    Advocan has also set its priorities for the year, which will be focussed 
    around indigenous students, mental health, international students, and 
    the financial aid system. The goal of the last priority is to re-work the 
    available aid money to make it more accessible. 

    Finally, there has been continuing work on course evaluations. For context,
    a committee of Senate has been meeting to discuss how course evaluations 
    happen for the last three years. After these three years, they have 
    submitted a report to the provost that was met with heavy opposition 
    from faculty. Professors seem to think that student feedback is not a 
    valid way to judge teaching. A more formal report will be going to senate 
    soon, and the \vpe\ is expecting it to be a lively meeting. 
    Student senators will be meeting to talk through the report and ensure that
    students are well informed to debate it on the sentate floor. The \vpe\ can 
    provide the report to councillors who are interested. 

    A councillor asked a question about international tuition hikes, and 
    whether that was a federal concern. The \vpe\ responded that it is, and 
    that the main concern is (in addition to the high prices), that it is very 
    in-consistent and that students should have stable costs. Another concern 
    is what extra services international students recieve in exchange for the 
    high prices they pay, and how many services are inaccessible to them. 
\end{information}

\section*{Business Arising from the Minutes}

\heading{Ratification of at-large members to committees}

\begin{motion}
    \birt Council ratifies Vaishnavy Gupta as an at-large member of the 
    Presidents Advisory Committee, and 
    \bifrt Council ratifies Hannah Sesink as an at- large member of the 
    Internal Administration Committee.
    \movers{\brian}{\tomson}

    \carries unanimously. 
\end{motion}

\section*{General Orders}

\heading{Fall Reading Break}
\begin{information}

After advocacy from the \vpe\ and a referendum of students, a pilot of a fall
reading break was started. The implementation of the break is now being 
reviewed, and a committee was created to this purpose.  

The results from surveys show that stakeholders were generally disatisfied. 
It seems inevitable that we will get a 1 week break at some point, but 
there is currently no clarity on how that will be implemented. 

The \vpe\ would like councillors to poll their constituents and determine:
\begin{itemize}
    \item Are students in support of a full week fall reading break, as a pilot? 
    \item What are the things we should give up in order to make that happen? 
\end{itemize}

The \vpe\ expresed that first years of the class of 2021 got short changed 
with orientation, and doesn't think we should cut any more into orientation. 

A councillor inquired about how other schools in the province have solved this 
problem. McMaster pushed orientation earlier in the year. Some councillors 
expressed support for that, but most schools don't go this route.  Many 
schools start classes on wednesday, which is undesirable since it cuts futher
into orientation week. 

The main options being considered are:
\begin{itemize}
    \item introducing sunday exams, and/or 
    \item removing one study day.
\end{itemize}

In terms of the purpose of the break, there is a conversation to make the week
more academic. The \vpe\ is looking for there to be an explicit list on what 
the break is a break from. They are looking for feedback specifically on 
what things students should be expected to be doing  during the break (i.e 
assignments, coop interviews, checking email, etc).

The \vpe\ took some in-person comments, but encouraged councillors ot email. 

A councillor expressed that coop interviews were totally unreasonable to happen
during a break, and that minor assignments over the break were fine, but there
should be no deadlines during or immediately after (since
services/assistance is not accessible during the break).  

The \vpe\ wrapped up that this was ongoing, and that he would appreciate
emails on the subject. He also would 
like everyoen to contact their societies and students about what they would
like fall breaks to look like, particularly what teh expection of students
should be over the break. 

\end{information}
    
\heading{Waterloo Nanotechnology Conference}

\begin{information}

    The Speaker read out a blurb from the Waterloo Nanotechnology Club (WNC). 

    A councillor asked about what kind of support Feds was giving WNC. They 
    were informed that WNC is pursuing funding through the internal funding 
    committee. If they are looking for more support they can reach out to 
    the \vpof\ or \vpi. 
\end{information}

\section*{New Business}

\heading{Appointment to PPC and IAC}

\begin{motion}
    \birt Council appoints Antonio Clarke to the at-large seat on the Policy
    and Procedures Committee and the Internal Administration Committee. 
    \movers{\brian}{\tomson}

    \carries unanimously.
\end{motion}

\heading{Committee Inquirees} 
\begin{information}

    A Councillor inquired about when the Education Advisory Committee and the
    Goverment Affiars Advisory Committee would be meeting. The \vped\ responded
    that they were looking to make those committees more functional, as in the
    past those committees had been a token gesture. A proposal for potential
    changes will be coming to those committee before it is brought to Council. 

    Another Councillor wished to know the status of the investigation into the 
    functionality of General Meetings, and the push to implement online voting.
    
    Some Directors in the room gave a brief update on the legal framework behind
    it, and encouraged councilors to modify or write a policy expressing
    Council's support for online participation in General Meetings. 

\end{information}

\heading{Meeting Times} 
\begin{information}

    A discussion on the date and time of Council Meetings began, 
    and the possibility of having them earlier on Sunday or moved to another
    day. It was determined that the Speaker would send out a poll
    to determine the best date and time for future meetings. 

\end{information}


\heading{Meeting Frequency} 
\begin{information}
    Council discussed the possibility of having meetings less than monthly. 
    Including the possibility of cancelling the August meeting. 

    The \vpof\ informed Council that the budget would need to be approved by 
    August, but that Council could delegate approval of it's section of the
    budget to be approved by the Board instead. 

\end{information}

\heading{Meetinig Dates} 
\begin{motion}
    \birt Council task the Speaker to send out a doodle poll to select the 
    meeting dates for the fall. 
    \movers{\brian}{\jill} 

    \carries unanimously. 
\end{motion}

\heading{Cancelation of the August Meeting}
\begin{motion}
    \birt Council designate the power to approve its portion of the budget to
    the Board, subject to it being brought back for information at the next
    Council meeting, and
    \bifrt Council allow the Speaker to cancel the August meeting if there
    is not enough agenda items.

    \movers{\brian}{\tomson}

    The \vpof\ stated that they will provide a session to allow Councillors to 
    ask questions in person

    \carries unanimously. \jason\ abstains.
\end{information}

