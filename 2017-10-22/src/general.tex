\section*{Report of the Chair}

\section*{Executive Reports}

Please see the attached written reports for the full reports from the Executive
to Student's Council. 

\heading{Report of the \vpe}
\begin{information}

    The \vpe\ reported that a few more students can sit on the student advisory
    panel on the coop fee review, so if any councillors or students 
    are interested than they can contact andrew to get a seat.

    He also announced that the director of CECA is retiring; she's been great 
    for campus in getting the fee review and other coop initiatives started. 
    There will be a transition as she leaves, and they are aiming for it to be
    non-disruptive.

    Another development in Waterloo that was not ready for student move-ins,
    making it the 4th in recent history. The \vpe\ is trying to get
    some municipal and provincial laws implemented that will prevent housing
    companies from leasing to students prior to the development being approved. 
    This will be a government sponsored document that will be mandated for 
    landlords to present to tennants, so that tennants know when their rights 
    are being abused. Will likely include language such that if a development 
    isn't ready a tennant won't be bound to stay.

    Regarding fall break, the \vped\ talked with societies and EngSoc. See 
    email update to Council from \andrewc\, he is working on a proposal to 
    have a week break, but this should not infringe any further into 
    orientation. 

    A councillor asked why the template existed if laws still took precedence,  
    specifically how does the contract solve move-in issues. 
    The \vpe\ explained that the template outlines the scope of the terms
    that are allowed to be in a lease agreement, to avoid illegal terms.
    Pre-tenancy rights are maybe not going to be included in the template, but 
    the hope is that if there are exact terms in an agreement there will be
    wording around what is allowed. The focus is mainly trying to put it on the
    governments radar. 

    A councillor asked if there was a system to prevent landlords from
    rejecting tennants if they asked for the government lease. The \vpe\
    responded that there would likely be bylaw fines if landlords were not
    providing the government lease. 

    A councillor asked how students will know about this, given that they don't
    know about the current laws. The \vpe\ responded that it will largely fall
    on students to educate themselves and one another. The major benefit is
    that the onus shifts to the landlords to educate tennants as to why their
    leases are legal. 

    There is currently also discussion about whether they want to email all
    first years about hosuing rghts, and the role of Feds in educating students
    about housing; as opposed to uWaterloo housing and residence. 

    The \vped\ is also prodding the region to get more general student 
    education to all the students in the city. 
\end{information}

\heading{Report of the \vpi}
\begin{information}
    Not present.
\end{information}

\heading{Report of the \vpof}
\begin{information}
    Not present.
\end{information}

\heading{Report of the \pres}
\begin{information}

    The \pres\ reported that there will be a progress report on the mental
    health panel on October 24th during the panel symposium. This will be a
    kind of mid-term report. This report will be livestreamed, and there is
    a QA panel happening right after the Feds GM, with online engagmeent through
    Twitter and Facebook. The panel will include the campus director of mental 
    health and the associate provost of students will moderate. \antonio\ and 
    Feridun will have opening remarks.

    The October General Meeting has been planned and organized, and members 
    have submitted proposals. Board approved the agenda which has been
    distributed in an email to all students.

    Federal advocacy week is coming up soon, and \antonio\ and \andrewc\ will 
    be in Ottawa advocating on national issues that affect students at 
    Waterloo. This will include issues like: student loans, indigenous student
    rights, and international student rights, among others.  

    Policy and procedures committee has been working with \alex\ to refine the
    freedom of expression policy that's on the agenda today. They are also
    working on an equity policy, and \antonio\ is communicating with the campus 
    director of equity to ensure all the bases are covered. The final policy
    will likely emalgamate several current policies.  Finally, PPC is working 
    on a long term change to the policy development process to facilitate
    creating and using policy more effectively.

\end{information}

\section*{Councillor Reports}

\seneca\ reported that he had run some surveys on environmental 
sustainability, particularly regarding the question of plastic water 
bottles. He also ran a survey regarding the proposal on freedom of
expression. \brian\ asked about online reimbursement last meeting,
and \seneca\ reported that 64\% of students polled were in favour of
online reimbursement.

\jason\ also gave a brief update: he will not be pursuing removing water 
at the Council level, but instead is working with the \vpof. Bottled water 
is being removed and boxed water is being sold instead, in a non-prominent 
location. 

\section*{Consent}

\heading{Approval of the minutes}
\begin{motion}
    \birt\ Board approve the minutes from September 17, 2017.
    \movers{\seneca}{\tristan}

    \carries\ unanimously.
\end{motion}

\section*{Regular}
\heading{Education Advisory Committee and Government Affairs Advisory Committee}
\begin{motion}
    \birt\ Council dissolve the Government Affairs Advisory Committee (GAAC); and
    \bifrt\ Council dissolve the Education Advisory Committee (EAC); and
    \bifrt\ Council dissolve the Entrepreneurship Advisory Committee; and
    \bifrt\ Council create the Education Advisory Committee, according to the
    attached terms of reference. 
    \movers{\andrewc}{\jason}

    Historically, EAC and GAAC were one committee. This committee was split to
    have more specific discussions, but unfortunately ended up creating an 
    environment where members were expected to have large amounts of context on 
    policy issues. This made councillors ineffective on these committees, and 
    the committees devolved into update committees. This is an attempt to 
    bring back a structure that allows councillors to have a larger impact
    on the lobbying decisions being made.

    \seneca\ and \jason\ (friendly) move to amend the motion to allow PPC to 
    clean up the policy with grammar and structure changes. 

    \carries\ unanimously.
\end{motion}

\heading{Freedom of Speech on Campus Policy}
\begin{motion}
    \birt\ Council adopt the policy on Freedom of Speech and Expression on
    Campus as presented, and 
    \bifrt\ Council task the President to work with the University to ensure
    the University endorses the Chicago Principles by the end of April 2018;
    and 
    \bifrt\ Council task the Executive to work with the University to ensure
    the University does not obligate clubs to hire security for their events;
    and 
    \bifrt\ that Council tasks the Internal Administration Committee to amend
    Club Procedure to ensure that clubs themselves are not liable for costs of
    event security until such time as the University drafts policy on this
    matter.  
    \movers{\alexander}{\seneca}

    The policy presented was refined with the help of PPC, and this version is 
    much clearer and in line with other Feds policy. \alexander\ ran through 
    the terms of the policy.  

    A councillor asked about the equity policy that was mentioned, \antonio\ 
    updated Council on the progress of the policy. 

    \begin{motion}
        \birt\ Council split the clauses into seperate motions. 
        \movers{\jason}{\andrewc}

        \jason\ expressed concern with the last item, as well as the provision
        stipulating that clubs should not have to pay for security if there is
        a cost associated with it. 

        The main intent with the grouping was to avoid connecting the policy
        with current campus lobbying. 

        \andrewc\, \jason\ motioned to vote on the first three clauses, and the
        last item seperatly. 

        \carries\ unanimously.
    \end{motion}

    Motion now reads:
    \begin{motion}
        \birt\ Council adopt the policy on Freedom of Speech and Expression on
        Campus as presented, and
        \bifrt\ Council task the President to work with the University to
        ensure the University endorses the Chicago Principles by the end of
        April 2018; and
        \bifrt\ Council task the Executive to work with the University to
        ensure the University does not obligate clubs to hire security for
        their events.
    \end{motion}

    \carries\ unanimously.
\end{motion}

\heading{Freedom of Speech on Campus Policy, part 2}
\begin{motion}
    \birt\ that Council tasks the Internal Administration
    Committee to amend Club Procedure to ensure that clubs themselves are not
    liable for costs of event security until such time as the University drafts
    policy on this matter.
    \movers{\alexander}{\seneca}

    There was some comments that clubs and Feds are the same pool of money, so 
    either way Feds is paying. IAC is empowered to allow clubs to spend money
    in any way they deem appropriate.

    A councillor brought up that the wordingo of this motion is pretty vague, 
    and that refering it to a committee is a good way for it to never be seen 
    again.

    \begin{motion}
        \birt\ Council postpone the motion till the next meeting.
        \movers{\andrewc}{\alexander}

        \carries\ unanimously. \seneca\ abstained. 
    \end{motion}
\end{motion}

\heading{Indigenous Engagement and Inclusivity}
\begin{information}
    Waterloo is at the industry standard on indegenous issues, and has only 
    started to address the issues. Feds has put together a report to help
    inform that discussion. 

    A councillor agreed with most of the principles of the policy, but had 
    reservations with the clause at the top of the second page:

    \begin{information}
        Be it further resolved that the Federation of Students shall advocate for the
        hiring of tenure-track Indigenous faculty members and professional and
        administrative staff across various disciplines;
    \end{information}

    The councillor is concerned that this created the impression that Feds was
    promoting hiring people for their ethnicity as opposed to their
    qualifications. 

    The \vped\ inquired about what specifically was the issue, as the purpose
    was to mainly address parts of the university community where having
    someone with knowledge of indigenous communities would be valuable.

    The \pres\ suggested ``removing barries'' as opposed to ``advocate for 
    the hiring of''.

    A few councillors expressed support for the original wording.

    A councillor inquired about operational concerns, and wanted more
    consultation with non-indeginious students.

    A councillor re-iterated their support for re-wording the clause on hiring. 

    A straw poll was held to inform how the \vpe\ further develops this clause.  

    A councillor expressed great support for the major and minor, and provided 
    an anecdote about the wait list for Mohawk language courses.
\end{information}

\heading{Fall Reading Break Proposal}
\begin{motion}
    \birt\ Council delegate the resolution and all relevant information
    regarding the Fall Reading Break proposal to the portfolio of the Vice
    President, Education, in consultation with the Policy and Procedures
    Committee.
    \bifrt\ Council empowers any Councillors who have opinions or feedback on
    the proposal to share these views directly with the Vice President,
    Education.
    \movers{\seneca}{\andrewc}

    \andrewc\ noted that orientation was not being targeted, and reiterated
    that this people should talk to the \vped. \andrewc\ emphasized that this
    this issue would be coming to the Education Advisory Committee for
    discussion.

    \carries\ unanimously.
\end{motion}

\section*{Other Business}

\heading{Ratification of \ousa\ Delegates}

\begin{motion}
    \birt\ Council ratify the Fall 2017 OUSA General Assembly following
    delegates recommended by the selection committee:
    \begin{itemize}
        \item Antonio Brieva,
        \item Andrew Clubine,
        \item Steven Jia,
        \item Alexander Wray,
        \item Hannah Beckett,
        \item Matthew Gerrits,
        \item Connor Plante,
        \item Daihane Zhu,
        \item Marcus Abramovitch, and
        \item two others to be appointed by the selection committee.
    \end{itemize}
    The selection committee was comprised of the \vpe, \pres, and 
    \printSpeaker.
    \movers{\andrewc}{\benjamin}

    \andrewc\ gave a brief introduction on what \ousa\ is to the counicllors,
    as well as encouraging students to email him if they wanted to be a 
    delegate, or if they just want to stop by. The General Assembly will be at 
    Laurier, next weekend.

    \carries\ unanimously.
\end{motion}

\heading{Commitee Membership}
\begin{motion}
    \birt\ the members of the previous EAC and GAAC committees become members
    of the new EAC.
    \movers{\jason}{\andrewc}

    \carries\ unanimously. 1 abstention.
\end{motion}

\heading{WatCard/UPASS Issues}
\begin{information}
    \stephanie\ expressed dissapointment with the confusion that is being created
    with the new WatCard/UPASS rollout. There is a lot of ``telephone talk'' as
    students attempt to navigate the situation. 

    The \vped\ responded that the UPASS is split between \vped\ and \vpof. 
    GRT hasn't decided exactly how the process is going to move ahead. It's 
    been confusing for Feds as well, because data is shared between the 
    university (WatCard office) and GRT; but, the agreement is between GRT and 
    Feds and doesn't involve the university. 

    The \vped\ announced that students will have an interim period where the
    card will be excepted without tap, and students will still have the leeway
    term (students pay for two terms, and the card works in the third if you
    are off-campus).

    The councillor re-iterated that there should be more comunication and that
    while recognizing the situation isn't entirely within Feds control, the 
    \vped\ should push GRT to be more transparent.

    The \pres\ indicated that communication with students saying that that they
    need to replace their WatCard was ongoing, and an email blast had been
    sent out. 

    It was also made clear that students will always have access to paying the
    UPASS fee as opposed to the GRT standard fare, even if they are off-campus.

    The \vped\ will go to negotiations with the feedback and will try to get
    more info up on the WatCard site and \url{feds.ca}, and will email 
    councillors with more information
\end{information}
