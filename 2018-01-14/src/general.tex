\section*{Consent}

\heading{Approval of the minutes}
\begin{motion}
    \birt\ Council approve the minutes from October 22, 2017.
    \bifrt\ Council approve the minutes from November 19, 2017.
    \movers{\seneca}{\lauren}

    \carries\ unanimously.
\end{motion}

\section*{Executive Reports}

Please see the attached written reports for the full reports from the Executive
to Student's Council. 

\heading{Report of the \pres}
\begin{information}

    The \pres\ reported that the President’s Advisory Committee on Student
    Mental Health had recieved a draft of the final report, and agreed that
    they were unsatisfied with it. They will be extending their deadline to
    create a final report to create more finalized recommendations and
    research. The \pres\ expressed that they would share the finalized report
    with Council as soon as it was publically available.

    Thy also hired the elections officer, and there is a motion to ratify them
    later in the agenda. There is also a new Feds Promotions Events 
    Coordinator, who is responsible for marketing Feds events, and will present
    at the all candidates meeting on ways to run a successful campaign, 
    including the effective use of social media. 

    The Policy and Procedures Committee (PPC) is continuing its work, Aisha has
    submitted a report for PPC that outlines some of the routes that Council 
    can take to facilitate the way that policy is created and reviewed. PPC 
    will hear this report at its next meeting, and will provide options to
    Council later in the term. 

    The General Meeting Task Force that was created at the last general meeting
    to investigate potential changes to general meetings is meeting for the 
    first time next week, and is aiming to have a presentation at the next
    general meeting. The committee will be deciding scope and structure of
    work.

\end{information}

\heading{Report of the \vpi}
\begin{information}

    The \vpi\ reported that the mental health partnership with Athletics
    regarding mental health is progressing well, they are looking for wellness
    ambasadors and to create a weekly program dedicated to mental and general
    health and well-being.  For more details, including long-term goals, see
    the \vpi's in written update. Both Campus Wellness and Athletics agree that
    this initiative needs to live on past the current Feds executives, and so
    will be managed by the special events coordinator.

    A Society Committee of President's (COPs) meeting occured and they have
    their first inter-society event booked. Moving forward, the \vpi\ is hoping
    to have a remote COPs meeting. 
    
    The \vpi\ is also working with the service advocacy directors to help them
    achieve their goals.

    Finally, the executives are creating remote office hours, in order to be 
    more available in-person at satalite campus's. This initiative may replace 
    Feds On Tour, and creates more consistent availability.
    
\end{information}

\heading{Report of the \vpof}
\begin{information}
    Not present.
\end{information}

\heading{Report of the \vpe}
\begin{information}
    
    The \vpi\ Worked on the \ousa\ TextBroke campaign, and while the formal 
    contest is over, the campaign continues as long as people are buying 
    textbooks. This marks the first outward step that \ousa\ has taken 
    regarding the next provincial election, and will be followed in March
    with a ``Get out the vote'' campaign. Feds will be sending out information
    about polling stations to students. 

    The \vpi\ reported that the fall reading break committee meetings keep 
    getting moved, but the next meeting will hopefully happen in February. 

    Finally, the \vpi\ will update Council on the current co-op fee situation
    later in the meeting.

    A councillor asked about e-registration to vote for students, and was
    wondering if there was any \ousa\ or Feds advertising for the
    e-registration. It is was stated that it is important for student unions to
    advertise e-registration, and advertising from both Feds and \ousa\ will
    largely be online and start in March.

    A councillor commented that there is \$5 million in the provincial budget 
    that people is being allocated by a vote, and is wondering if we are also
    advertising possible higher education uses for that money. The \vpe\ will
    be presenting to the house economic and finance committee about funding to
    universities not keeping up with inflation, among other topics. 

    A councillor suggested that Board should fill out the form for the
    provincial money, and promote it to students.

\end{information}

\section*{Report of the Speaker}

Science councillor Cameron Mills has resigned from his position effective
immediately. 

\tianqi\ was acclaimed to the Science Councillor seat.

\section*{Regular}
\heading{Updates on Presidential Advisory Committee – Student Mental Health:
Student Services Panel}
\begin{information}
    The Student services panel recommended that the University follow \ousa\
    and Feds' recomendations. The student services panel also recommended that
    Feds work to coallese the efforts of the student societies. 
\end{information}

\heading{Coop Fee Increases}
\begin{motion}
    \whereas\ the Federation of Student acknowledges the UW Staff Association’s
    Memorandum of Agreement: Staff Compensation 2015\-2018 which requested an
    increase in staff salaries,

    \whereas\ the failure to inform student representatives about fees
    increases until less than 24 hours prior to the Co-operative Education
    Council (CEC) meeting does not constitute a reasonable and expected level
    as transparency from a student-funded organization,

    \whereas\ failing to inform the general student body before moving the
    co-operative education fee increase to Board of Governors for approval
    further violates expected transparency,

    \whereas\ the proposed CECA fee increase is considerably higher than the
    Consumer Price Index adjustment and is primarily for salary increases,

    \birt\ the Federation of Students protests any increase co-operative
    education fees prior to the completion of the Co-op Fee Deep-Dive analysis
    investigating the need for, and impact of, any such increases; and

    \bifrt\ the Federation of Students requests a hiatus on any changes to co-op
    fees until the completion of the deep-dive investigation; and 

    \bifrt\ the Federation of Students requests decisions regarding changes in
    fees are conducted in a more transparent manner involving continuous
    dialogue with students (particularly through Co-op Student Council and
    Co-op Education Council); and 

    \bifrt\ fees increases borne by students above the annual Consumer Price
    Index adjustment should have approval-in-principle by elected student
    representatives.   
    \movers{\seneca}{\andrewc}

    \seneca\ and \andrewc\ spoke to the motion. \andrewc\ outlined that there
    were complicated issues with the review, and passed off his speaking term
    to \hannah, the Co-op Affairs Commisioner. 

    \coopcom\ expressed that the committee got off to a late start, and the
    staff member from co-op was pulled off after only a week or two due to a
    severe family emergency. Thus, the committee only started being productive
    in the fall term, due to staffing issues on CECA's side. The analysis of
    the current state took up most of the the fall term, and will be presented
    at the next meeting of the committee.

    The report is meant to show the flows of money, including how much staff
    cost in each area, what do they do, and how much money goes into each
    ``bucket'' of service that CECA performs. This includes things like
    faculty relations, employer relations, student experience, etc. 

    There is also an environment analysis, which involves looking at other
    schools. Many schools have larger coop fees with less services, or coop
    fees broken out by faculty. 

    This report will then go to an advisory committee of students that was
    struck last year, who can request deeper dives into particular cost areas. 

    CECA is putting their most senior staff on deep-dive and they are working
    around the clock. Staff in the department recognize that students are
    upset.   

    Ideally, the report on the curent state of the fee will go out to the
    student body this term. There is also a website that is tracking the
    progress and state of the project, as well as contact information so
    students can share feedback. This will likely be up in the next week or
    two.

    The \andrewc\ took back the floor, to explain that there was a good spirit 
    to the motion and touched on how the trust of the student body was really 
    important, and something that CECA took seriously. 

    A councillor voiced support for the motion, and talked about the 
    lack of transparency that was being seen from students. Stated that the 
    fee review project was great, but needs more of a stance from Feds. 

    \seneca\ pointed out that there were ``WHEREAS'' clauses in the motion to
    outline that they recognize the responsibility for CECA to increase
    salaries. Increases of the fee to account for salaries shouldn't happen
    if there is efficient staff management; the revenue goes up as their are 
    more students. Since students are obviously unhappy with how CECA works, why
    do they deserve salary increases? If there is an MOU with staff about 
    salary increases, than the institution should be paying the salary
    increases until they can provide better customer service. 

    A councillor echoed support for the motion, stating that if there is a deep
    dive happening then the perception of getting only 24 hours notice for a
    fee increase while a deep-dive is ongoing feels disingenous and ``throws
    the deep dive under the bus''. The expectation from students coming in is
    that they will have support from this department, but there is ongoing
    issues with students not getting help from advisors, from PD being
    unhelpful, among other issues.

    The \vped\ explained that the motion being presented to Council was about
    capacity and not service, since the fee review is meant to address the
    service. \coopcom\ re-iterated that the current concern is around the fee
    increase and transparency, not about the services being provided to
    students. Running the worlds largest co-op program will have fee increases
    while the review is happening. 

    It was stated that there is an issue on the campus with shouting into the
    void of the internet without escalating issues to the decision makers who
    can change things. \coopcom\ also expressed support for the idea that CECA
    was borderline negligent with the way that they proposed the fee increase,
    but the issue was complex and they were hamstrung by other university
    admins. 
    
    \seneca\ expressed his support for the points that \coopcom\ and the \vped\
    had said, and that stated that the motion does not tie the hands of the
    executives.

    Council was informed that the salary increase recommendation is a 2.5 year 
    recommendation, and has been known about for a long time. CECA should have
    seen this coming and been more transparent. \seneca\ expressed that while
    it wasn't gross incompetence, but there is an amount of budgetary
    negligence. 
    
    \seneca\ also expressed that there are great people at CECA, and stated
    for the record his opinion that ``Franco Solimano is a great person who 
    deserves a raise, and has spoken to societies and CEC about issues in 
    coop.'' He would like it noted that just because CECA is raising fees and
    there are transparency issues in the institution does not imply that 
    the people in the institution are ``bad''. The department wants to raise
    fees given a positive economy of scale, while CECA claims there is a
    negative economy of scale. Students are entitled to hear justification that 
    there is a negative economy of scale. 

    A councillor agreed with the mover, and stated that it was important for
    Council to have a stance on issues as they happen. They also agree that
    there is some nuance with the issue, and that Councillors have a 
    responsibility to work with CECA and to maintain a positive partnership
    with the university departments. Council needs to be cohesive with how they
    frame Feds messaging on the issue.

    \seneca\ mentioned the possibility of having a letter signed by the 
    executives, the speaker, and/or the secretary of Council that can be
    distributed to the student body. He also reiterated that while the fee
    increase motion would almost certainly pass the Universities Board of
    Governors, it was necessary to express that students were unhappy.

    \andrewc\ expressed that everyone should agree with the things that
    \seneca\ was saying, or there may be problems with constituent 
    representation. The issue comes when we try to understand the issues
    that are facing the internal departments of the university. We aren't 
    unionized, but we might as well be with the way salary negotiations happen
    on campus. The \vped\ is hesitant that the motion does not show that Feds 
    understands the binds that the university is under. The \vped\ expressed
    support for a letter coming from Council, that goes over the issues with 
    the process and the way that this fee change happened. They do not think 
    it's fair to be naive and expect a 0\% fee increase. 

    \seneca\ re-iterated support for a letter, but wants an official action
    from Council in addition to a letter. The process for this fee increase was
    awful; but that he is open to ammendments to the motion to express the
    complexity of the issue or to remove the clause that references the 
    Consumer Price Index. It is the resolve of Council to express what their
    candidates want, and he does not want to unnecessarily bind executives.

    Austin Richard (Acting President of the Science Society) sympathizes with
    the complexity of the issue, but expressed that it was tough to justify a
    fee increase when they didn't have a clear understanding of where the fee
    was going.

    \coopcom\ responded that in an ideal world the fee would be frozen while
    the fee review was happening, but the reality is that the costs would
    increase while the fee review was happening, and the fee review may result
    in completely different structure, services required, etc. The current
    process is what we have right now, but it is not too much of a stretch to
    imagine it will be completely different in a year.

    \vped\ expressed that student expectations have fundementally shifted over
    the last year and a half; it may be more valuable for Councillors to
    participate in the fee review, and asked councillors to share their
    expectations of how the process for the fee increase should happen.

    \begin{motion}
        \birt\ Council strike clauses 2 and 4 of the motion:

        \bifrt\ the Federation of Students requests a hiatus on any changes to
        co-op fees until the completion of the deep-dive investigation; and 

        \bifrt\ fee increases borne by students above the annual Consumer Price
        Index adjustment should have approval-in-principle by elected student
        representatives.   
        \movers{\jason}{\andrewc}

        \jason\ withdraws his ammendment.
    \end{motion}

    Council is considering the the original motion. 

    \begin{motion}
        \birt\ Council strike clause 4 of the motion, modify clause two to
        add ``requests to the Provost a'', and add a clause that requests the 
        creation of a letter explaining the situation.        
        \movers{\andrewc}{\seneca}

        The motion would read:

        \begin{motion}
            \birt\ the Federation of Students protests any increase in
            co-operative education fees prior to the completion of the Co-op
            Fee Deep-Dive analysis investigating the need for, and impact of,
            any such increases; and

            \bifrt\ the Federation of Students requests to the Provost a hiatus
            on any changes to co-op fees until the completion of the deep-dive
            investigation; and 

            \bifrt\ the Federation of Students requests decisions regarding
            changes in fees are conducted in a more transparent manner
            involving continuous dialogue with students (particularly through
            Co-op Student Council and Co-op Education Council); and 

            \bifrt\ Council, in conjunction with the Executive Board, draft a 
            letter outlining and explaining the events surrounding the fee 
            increase and the actions taken by Council.
        \end{motion}

        \carries\ as the motion is friendly.

    \end{motion}

    \andrewc\ asked for clarification on what the letter should include, he
    and \coopcom\ can likely create a letter in the next few days. He would not
    expect Council to all be editors on the letter, but Council can provide
    high level feedback on the letter. He expects to see changes in process for
    next terms fee.

    \andrewc\ is also hoping to have an update on the process for February or
    March, and reiterates that it would be very valuable for passionate 
    councillors to attend the review meetings.

    \carries\ unanimously.
\end{motion}

\heading{Elections and Referenda Officer} 
\begin{motion}
    \birt\ Council ratifies the appointment of the Elections and Referenda
    Officer, Rency Luan.
    \movers{\antonio}{\seneca}

    \carries\ unanimously.
\end{motion}

\heading{Election and Referenda Appeals Committee}
\begin{motion}
    \birt\ Council appoints \blank\, \blank\, and \blank\ to the Elections and
    Referenda Appeals Committee (ERAC).
    \movers{\antonio}{\seneca}

    It is customary for President and Speaker to be on this committee 
    if they are not running. \begin{itemize}
        \item \andrewc\ nominates \elizabeth\ \antonio. 
        \item \seneca\ nominates \rebecca. 
        \item \nickta\ nominates herself.
    \end{itemize}

    \nickta\ withdraws her nominations. 

    \elizabeth, \antonio, and \rebecca\  are acclaimed to the blanks.

    \carries\ unanimously.
\end{motion}

\heading{Student Life Committee Appointments}
\begin{motion}
    \birt\ Council appoints \blank\ and \blank\ to the Campus Life Advisory
    Committee (CLAC).
    \bifrt\ Council appoints \blank\ and \blank\ to the Internal Administration
    Committee (IAC).
    \bifrt\  Council appoints \blank\ to the Leadership Awards Committee (LAC).
    \movers{\jill}{\andrewc}

    Nominees for CLAC:\begin{itemize}
        \item \tomson\ nominates himself.
        \item \alexander\ nominates \seneca.
        \item \alexander\ nominates \nickta.
        \item \andrewc\ nominates \abigail.
    \end{itemize}

    Council heard the nominee's platforms.

    Ballots were collected for CLAC.\@

    \seneca\ and \tomson\ were appointed to CLAC.\@

    Nominees for IAC:\begin{itemize}
        \item \seneca\ nominates \alexander.
        \item \seneca\ nominates \jennifer.
    \end{itemize}

    Nominees were acclaimed.

    \alexander\ and \jennifer\ wee appointed to IAC.\@

    Nominations for LAC:\begin{itemize}
        \item \stephanie\ nominates herself.
        \item \tomson\ nominates himself.
        \item \subham\ nominates himself.
    \end{itemize}

    Council heard the nominee's platforms.

    Ballots were collected for LAC.\@

    \stephanie\ was appointed to LAC.\@

    The motion now reads:
    \begin{motion}
        \birt\ Council appoints \seneca\ and \tomson\ to the Campus Life
        Advisory Committee.
        \bifrt\ Council appoints \alexander\ and \jennifer\ to the Internal
        Administration Committee.
        \bifrt\ Council appoints \stephanie\ to the Leadership Awards
        Committee.
    \end{motion}

    \carries\ unanimously.
\end{motion}

\heading{Ratify on Service Coordinators}
\begin{motion}
    \birt\ Council approve the service coordinators for Winter 2018. 
    \movers{\jill}{\seneca}
    \carries\ unanimously.
\end{motion}

\heading{Indegenous Policy}
\begin{motion}
    \birt\ Council approves the Indigenous Policy as presented.
    \movers{\andrewc}{\seneca}

    \carries\ unanimously.
\end{motion}

\heading{MathNews}
\begin{motion}
    \birt\ Council postpones a in-depth discussion to the next meeting.
    \movers{\andrewc}{\seneca}

    \jill\ gave a high level overview of the issue. There was a meeting with
    the Feds executives, MathNews, and MathSoc; everyone is now on the same
    page. 
    
    They are still working on the MOU between MathSoc and MathNEWs, and 
    while Feds is not a signing authority, there is information on the MOU that
    needs to be corrected. 
    
    Feds has appologized for the miscommunications, and is moving forward with
    meetings to make sure that students are being well-served and that 
    these kinds of mis-communications do not happen again.
    
    \carries\ unanimously.
\end{motion}

\section*{Other Business}

\heading{Elections and Feds Email}
\begin{motion}
    \birt\ Council postpone the end of the nomination period for 48 hours. 
    \movers{\antonio}{\seneca}

    The Feds email servers have been down, and they have possibly missed some
    nominations. It was noted that societies who align their elections with the
    Feds elections should be notified. 

    A councillor asked if this would also affect the Senate elections. \pres\
    responded that they would inform the Secretariat and extend the nomination
    period. 

    \carries\ unanimously.
\end{motion}

\heading{Society President and Feds President}
\begin{information}
    A councillor brought up a situation where a student is planning on running
    for both society and Feds president, and would like affirmation that this
    is a conflict of interest and a student cannot hold both roles.

    Council agreed that this would be a conflict of interest.
\end{information}

\heading{Meeting Dates}
\begin{motion}
    \birt\ Council meet on February 11, 2018 and March 25, 2018. 
    \movers{\antonio}{\seneca}

    \carries\ unanimously.
\end{motion}
