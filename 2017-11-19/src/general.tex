\section*{Consent}

\heading{Approval of the minutes}
\begin{motion}
    \birt Board approve the minutes from May 23, 2017.
    \movers{\brian}{\abdullah}

    \carries unanimously.
\end{motion}

\section*{Report of the Chair}

\section*{Executive Reports}

Please see the attached written reports for the full reports from the Executive
to the Board of Directors. 

\heading{Report of the \pres}
\begin{information}
    225 ish people came to the GM. Task force  for GM's proposal that got 
    passed, board appointed their members. Engagment and structure of 
    general meetings, business, what is council's role, etc. 

    committee pannel on student mental health, tried to tie to general 
    meeting, but meeting ended early so not all 225 people attended, in
    person attendance was much smaller, about 30-60 people throughout. 
    Facebook live had close to 2000 people watching, and stayed for about
    the entire stream. Uni President was there, and stayed for most of teh
    panel. There was lots of good followup discussion from the panel.

    Antonio got emails from students, agreeing and disagreeing with some of the
    points that were made on teh panel. 

    thirdly, federal advocacy, antonio and andrew went to ottawa from nov 6-10 
    there were about 15 meetins throughout the week, included in the written
    report is a brief overview of the advocacy that occured. 

    % TODO get antonio's written report

    a councillor asked what may have caused teh higher attendance at the GM,
    antonio replied that it was likely partly due to it being earlier in 
    the day. 

    social media sharing also contributed to the turnout, councillors actively
    recruited people to come vote.

    also, the vpof mentioned that they streamlined the process to sign people
    in, so there was less of a line and they could start on-time.

\end{information}

\heading{Report of the \vpi}
\begin{information}
    highlighting on societies, COPS meeting every 3-4 weeks, almost a problem
    of over-engagement as president are bringing their vpinternals, 

    highlighting IT needs, events, collaboration between societies. 

    one-on-one meetings with societies to make sure they are adressing their 
    needs.

    wrap up week, trying to determine how to best meet students needs, especially
    mental health, meeting with athletics and counselling to make sure it is
    effective

    warrior tribe participated in santa clause parades

    bike centre is done renovations and is closing for winter

    volunteer appreciation happened, reached capacity. doing an evaluation to
    make sure the appreciation meets student needs

    meeting with women centre and mates advocacy directors to help them
    expand their contact lists and work on uni level advocacy

    days against gender based violence

    trial running sessions on how to run an effective meeting, and conflict management to
    better train student leaders on campus. 

    holding more satelite campus office hours
    
\end{information}

\heading{Report of the \vpe}
\begin{information}
    coop fee review is on pause, some questions from engineering due to
    A vs B societies and how to handle consulting. Andrew will inform council
    and VPed's of societies, and will provide resources to have broad consultation
    but it will be up to societies and councillors to figure out what that
    looks like in their faculy. 

    10 policy papaers at OUSA, system vision, accountability, open education,
    open liscencing, indigenous students. all passed, currently being edited
    and posted on ousa's website. 

    also launched a publication, in it together, 10-12 page report built in
    collaboration wiht primary stakeholder in PSE sector, first time these
    groups have come together on. About mental health. Largest turnout for a
    7:30 AM reception. a day of lobbying , very well recieved. Government
    seems to be aware that they should be handling mental health as a structura
    issue and not just a funding one.

    New group for advocacy, relatively new, seems to have high impact so far. 
    New OSG style federal thing. Tax credits -> grants. not appealing in
    terms of political movement, but everyone suggested this would be good
    moving forward, so hoping for some thing in the next few years. 

    policy 70-71-72, university policies on when things go wrong for students.
    they are a mess, and are being reviewed, will likely bleed into next
    spring and the committee is great, starting from a balnk slate. 

    ousa is moving into open educational resources, so is campus bookstore, 
    basically open resources and especially textbooks will be a really great
    resource for students. only cost is "do you want to print this". faculty
    can edit, write, modify to fit the context. OuSA will run a campaign in
    January, Feds will be mirroring, mostly a social media campaign, but FEds
    would like to do a letter-writing campaign towards faculty. IF councillors
    would like to get involved they can do so.

    a councillor asked if there was a place or repeort to find what was passed
    at ousa. 
    the vped, directed councillors to check the ousa website, as all of the 
    policies would be posted by the end of the month.

    there is an ousa survey out right now, might be extended, since other
    school's ethics boards took a while to approve it. survey done 
    every two years to gauge student expeirence on campus. 

\end{information}


\heading{Report of the \vpof}
\begin{information}

    The secretary complimented teh vpof's mustache. 

    feedback from student groups, particularly societies, and employers doing
    catering in the bomer, they realize that most catering food is fried. new
    catering menu that is more varied, and less fried. 

    bottled water is being fazed out foro 500ml and 1L boxed water options

    SLC campus bubble and things will need to close as window beams are removed
    due to construction

    commuity kitchen is doing well, seems to have good uptake on student groups
    making things there for their events. 

    legal survey will now not be going out in november, strategic decision to 
    move it to the beginning of january. survey is done and ready to launch.

    job fair planned for the beginning of january for all of the part time 
    positions in feds, and they can do quick interviews and ask questions
    about the positions (professional development, responsibilities, etC). 
    Was done in response to a questionaire from the last open house. Response
    was about 80\% to knwo more about feds part time jobs. Planned virtual 
    options as well, maybe to integrate key other part time options on campus.

    A student asked if there would be an information campaign prior to the 
    legal survey, so student knew what they were being surveyed on. 

    There will be a brief information shared wth the survey, and there will be
    some social media information shared. 

\end{information}

\section*{Regular}
\heading{Item 1}
\begin{motion}
    \birt\ council appoints \blank\ and \blank\ to the two council seats on the
    Task Force looking into the structure and engagement at general meetings. 
    \movers{\antonio}{\brian}

    Timeline will likely a bi-weekly meeting, an hour to an hour and a half.

    \alexander\ nominates \senecca. 
    \jason\ nominates himself.
    \alexander\ nominates himself.
    \andrewc\ nonminates \ben.

    There will be an election. Each candidate gave a brief overview of their
    goals and why they should be appointed to this committee.

    Councillors voted, two nominees per ballet.
    The vote was counted. Alexander and Ben were appointed ot the blanks. 
    The motion now reads:

    % TODO re-write motion

    \carries\ unanimously.
\end{motion}

\heading{Election Dates}
\begin{motion}
    \birt\ Council set the nomination period for Feds' elections to start 
    November 20 and end January 15; 
    \bifrt\ Council set the campaign period to begin January 22, and 
    end February 7 at 10PM.%
    \movers{\antonio}{\senecca}

    \carries\ unanimously.
\end{motion}

\begin{motion}
    \birt\ Council destroysthe ballets from the previous election on 
    the General Meeting Task Force. 
    \movers{\brian}{\jason}

    \carries\ unanimously.
\end{motion}

\section*{Other Business}

\heading{Misinformation on Feds Website}
\begin{information}
    A councillor inquired about potential mis-information in the feds reporting
    on the online voting motion of the General Meeting
\end{information}

\heading{Information Reporting Regarding University Departments}
\begin{information}
    A councillor shared that someone in Science had recieved almost 1000 complaints
    with coop from students, and was wondering how these complaints should be 
    shared. The councillor also sugested that Feds create a ``complaints form''
    on the website so that students can submit complaints against the university.

    The vped replied that yes, the focus wa son the fee review, and there is now
    a reorg that makes CECA under the provost instead of an arms-length body. 
    Hoping that coop department and experienctial education are all working for 
    students. 

    Councillors expressed support for more centralized reporting. 
\end{information}

\heading{Student Reprentatives Logging Hours}
\begin{information}
    SciSoc asked Science Councillors to ask Feds Student Councillors, 
    Executives, and Directors to log their hours. 

    Coucillors expressed that self-reporting is often inaccurate. Executives
    reporitng is under the purview of board. 

    Councillors generally feel pretty neutral on the idea of self-reporting 
    their own hours. 
\end{information}

\heading{OUSA Blog}
\begin{information}
    The \vped\ invited councillors to check out the topics posted about on the
    blog, and invited Councillors to write for the blog. 
\end{information}

\heading{OUSA Blog}
\begin{information}
    The \vped\ invited councillors to check out the topics posted about on the
    blog, and invited Councillors to write for the blog. 
\end{information}

\heading{January Meeting Date}
\begin{motion}
    \birt\ Council meet on January 14, 2017 at the regular time. 
    \movers{\jason}{\brian}
    \carries\ uanimously.
\end{motion}

\heading{Student Funding Information}
\begin{information}
    Councillors would like to see more communication and outreach for the 
    various sources of funding that students have available to hold events
    and improve student life. 

    Also councillors would like to see an investigation on speeding up the
    process for a group to get money, and more awareness so that Feds clubs
    aren't always reaching out to external sources like Faculty Endowment 
    Funds. 

    The \vpi\ responded that they are currently talking about how to advertise
    and market the Internal Funding Committee (IFC) and relevant funds. 
    
    Currently IFC meets bi-weekly, and proposals almost always came in on an
    off week.

    Most proposals came with less than 2 weeks notice, which is very difficult
    for students to work with as a time-frame. 

    In terms of Student Life Endowment Fund, funding events doesn't come out
    of that fund, and it has historically been almost exclusively for investing 
    in student lounges.

    A councillor also asked if it would make sense for Faculty Endowment Funds
    to sit together as a committee. Councillors agreed this would be great,
    but would be up to those organiations. 

    Councillors also expressed wanting to see more information on Faculty
    Funds on the website.
    
\end{information}

\heading{}
\begin{information}
    \birt\ Council cancel its December 3rd meeting.
    \movers{\elizebeth}{\seneca}

    Councillors expressed that Council could vote on the 
    Inigenous Student Policy over email if we needed to get that through 
    early.

    A straw poll was held on whether to have a straw poll.

    \carries\

    \jason, \alexander, \stephanie, noted against.
\end{information}
