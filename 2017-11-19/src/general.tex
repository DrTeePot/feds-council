\section*{Consent}

\heading{Approval of the minutes}
\begin{motion}
    \birt\ Council approve the minutes from October 22, 2017.

    \begin{motion}
        \birt\ Council table the minutes from October until the next meeting.
        \movers{\tristan}{\brian}
        \carries\ unanimously.
    \end{motion}
\end{motion}

\section*{Executive Reports}

Please see the attached written reports for the full reports from the Executive
to Student's Council. 

\heading{Report of the \pres}
\begin{information}
    The \pres\ reported that approximately 225 people came to the General 
    Meeting, and that the Task Force for GM Engagement was created; the Board
    has appointed their members, Council will appoint members this meeting.
    
    The committee pannel on student mental health that was meant to happen
    after the General Meeting happened, but the General Meeting ended early so 
    in-person attendance was lower than expected \emdash about 30 to 60 
    people throughout. However, Facebook live recorded almost 2000 people 
    watching, and those people stayed for almost the entire stream. The
    University President also attended, and stayed for most of the
    panel. The \pres\ noted that there was a good followup discussion from the 
    panel, including emails from students who agreed and/or disagreed with some
    of the points that were made on the panel.

    Thirdly, \antonio\ and \andrewc\ went to Ottawa from November 6th to the 
    10th, and attended about 15 meetings throughout the week. A brief overview
    of the advocacy that occured is included in the written report. 

    A councillor asked what may have caused the higher attendance at the GM,
    the \pres\ replied that it was one reason was likely the timing of the
    meeting being earlier in the day. It was also hypothesized that social
    media sharing also contributed to the turnout, as councillors actively
    recruited people to come vote. The \vpof\ mentioned that they streamlined
    the process to sign people in, so there was also less of a line and the
    meeting could start on-time.

\end{information}

\heading{Report of the \vpi}
\begin{information}

    The \vpi\ reported that the Committee of Society President's was meeting 
    every 3 to 4 weeks, and was almost having a problem of over-engagement as 
    President's brought additional Vice President's. Some of the biggest things
    that societies needed was IT support, events support, and greater
    collaboration between societies. 

    The \vpi\ has also been having one-on-one meetings with societies to make
    sure Feds is adressing their needs.

    Student mental health continues to be a priority, as the \vpi\ does 
    research on how to implement Wrap up week to best meet students needs. They
    are meeting with athletics and counselling to make sure the week is
    effective.

    Volunteer appreciation reached capacity this year, and the \vpi\ is 
    conducting an evaluation to make sure the appreciation event meets student
    needs.

    There is continued work on centralizing the advocacy that services are 
    doing. Particularly, the \vpi\ is meeting with the advocacy directors of 
    The Women's Centre and MATES to help them expand their contact lists and 
    expand their work at the university level. 

    Feds is also running trial sessions on how to run an effective meeting, as
    well as conflict management sessiomn to better train student leaders on
    campus. 

    Finally, the \vpi\ is holding more satelite campus office hours.
    
\end{information}

\heading{Report of the \vpe}
\begin{information}

    The \vpe\ reported that the co-op fee review is currently paused. There
    were some questions from Engineering due to A vs B Engineering Societies and 
    how to handle consulting. Andrew will inform council and the Vice President
    Education/Academic of each society, and will provide resources to have
    broad consultation; however, it will be up to societies and councillors to
    figure out what that looks like in their faculy. 

    There were 10 policy papers at \ousa\ including: 
    \begin{itemize}
        \item system vision, 
        \item accountability,
        \item open education,
        \item open licensing, and
        \item Indigenous students.
    \end{itemize} They are currently being edited and will be posted on \ousa's
    website. 

    \ousa\ also launched a publication, ``In It Together'', a 10 to 12 page 
    report built in collaboration with primary stakeholders in the 
    post-secondary education sector. This is the first time these
    groups have come together on a project. The publication will be
    predominantly about mental health. When it launched, it was said to have
    had the largest turnout for a 7:30 AM reception, which was followed by  a 
    day of very well recieved lobbying. The government seems to be aware that 
    they should be handling mental health as a structural issue and not just a
    funding one.

    They have also created a new group for advocacy, which so far seems to have
    high impact. They are currently working on a new OSG-style (moving tax
    credits into up-front grants) federal funding structure. This is currently
    not appealing in terms of political movement, but everyone has suggested 
    this would be good moving forward; the \vped\ hopes for some greater 
    progress in the next few years. 

    A review of policies 70, 71, and 72 is currently ongoing. These are the 
    policies that outline what happens when things go wrong for students.
    This review will likely bleed into next spring, but the committee is doing
    really well. They have decided to start from a blank slate. 

    \ousa\ is moving into open educational resources, along with the campus 
    bookstore. These are basically open resources, especially textbooks, that
    will be a great resource for students. The  only cost will be if students
    want hard-copy versions of the resource. Faculty can edit, write, modify 
    the content to fit the context. \ousa\ will be running a (mostly a social 
    media) campaign in January, which Feds will be mirroring, but Feds would 
    like to also do a letter-writing campaign towards faculty. If councillors
    would like to get involved they should contact the \vpe.

    A councillor asked if there was a place or report to find what had been 
    passed at \ousa. The \vped\ directed councillors to check the \ousa\ 
    website, as all of the policies would be posted by the end of the month.

    Finally, the \vped\ informed Council that there was an \ousa\ survey out 
    right now, which might be extended since other school's ethics boards took 
    a while to approve it. This survey is done every two years to gauge 
    student's experiences on campus. 

\end{information}

\heading{Report of the \vpof}
\begin{information}

    The secretary complimented the \vpof's mustache. 

    Based on feedback from student groups, particularly societies and employers
    doing catering in the Bomber, they have implemented a new catering menu
    that is more varied, and less fried. 

    Bottled water is being fazed out by 500ml and 1L boxed water options, 
    presented in non-prominent locations.

    SLC Campus Bubble will need to close as window beams are removed due to 
    construction.

    Commuity Kitchen is doing well, and seems to have good uptake from student 
    groups cooking and baking for their events. 

    The student legal survey will not be going out in November as previously 
    communicated. There was a strategic decision to move it to the beginning 
    of January. The survey is done and ready to launch.

    A part-time job fair is planned for the beginning of January, and they are
    planning to do quick interviews and field questions about the open 
    positions for students (professional development opportunities, 
    responsibilities, etc). This was done in response to a questionaire from 
    the last open house. 80\% of respondants wanted to know more about Feds
    part time jobs. They are also planning to have virtual options, and are
    considering integrating other part-time options on campus.

    A student asked if there would be an information campaign prior to the 
    legal survey, so students were aware what they were being surveyed on. The
    \vped\ responded that there will be some brief information shared with the
    survey, and there will be some social media information shared. 

\end{information}

\section*{Regular}
\heading{Task Force Appointments}
\begin{motion}
    \birt\ council appoints \blank\ and \blank\ to the two council seats on the
    Taskforce on General Meeting Engagement. 
    \movers{\antonio}{\brian}

    The taskforce will be looking into the structure and engagement at general
    meetings. Timeline will likely be a bi-weekly meeting, for an hour to an
    hour and a half.

    \alexander\ nominates \seneca. 
    \jason\ nominates himself.
    \alexander\ nominates himself.
    \andrewc\ nonminates \ben.

    There will be an election. Each candidate gave a brief overview of their
    goals and why they should be appointed to this committee.

    Councillors voted, two nominees per ballet.
    The vote was counted. \seneca\ and \ben\ were appointed to the blanks. 
    The motion now reads:

    \begin{motion}
        \birt\ council appoints \seneca\ and \ben\ to the two council seats
        on the Taskforce on General Meeting Engagement. 
    \end{motion}

    \carries\ unanimously.
\end{motion}

\heading{Election Dates}
\begin{motion}
    \birt\ Council set the nomination period for Feds' elections to start 
    November 20 and end January 15; 
    \bifrt\ Council set the campaign period to begin January 22, and 
    end February 7 at 10PM.%
    \movers{\antonio}{\seneca}

    \carries\ unanimously.
\end{motion}

\heading{Destruction of Ballets}
\begin{motion}
    \birt\ Council destroys the ballets from the previous election on the
    General Meeting Task Force. 
    \movers{\brian}{\jason}

    \carries\ unanimously.
\end{motion}

\section*{Other Business}

\heading{Misinformation on Feds Website}
\begin{information}
    A councillor inquired about potential mis-information in the Feds reporting
    on the online voting motion of the General Meeting.
\end{information}

\heading{Information Reporting Regarding University Departments}
\begin{information}
    A councillor shared that someone in Science had recieved almost 1000
    complaints with coop from students, and was wondering how these complaints
    should be shared. The councillor also sugested that Feds create a
    ``complaints form'' on the website so that students can submit complaints
    against the university.

    The \vped\ replied complaints could be forward to themselves or Franco,
    that there was a focus in the coop department on the fee review, and that
    there is currently a reorganization of the department that defines CECA
    under the Provost instead of as an arms-length body. They are hoping that
    having a seperate departments under the Provost will be a structure that
    works better for students. 

    Councillors expressed support for more centralized reporting within Feds. 
\end{information}

\heading{Student Reprentatives Logging Hours}
\begin{information}
    SciSoc requested Science Councillors to ask Feds Student Councillors, 
    Executives, and Directors to log their hours. 

    Coucillors expressed that self-reporting is often inaccurate. Executive
    reporting is under the purview of board. 

    Councillors generally feel pretty neutral on the idea of self-reporting 
    their own hours. 
\end{information}

\heading{OUSA Blog}
\begin{information}
    The \vped\ invited councillors to check out the topics posted on the \ousa\
    blog, and invited Councillors to write for the blog. 
\end{information}

\heading{January Meeting Date}
\begin{motion}
    \birt\ Council meet on January 14, 2017 at the regular time. 
    \movers{\jason}{\brian}
    \carries\ uanimously.
\end{motion}

\heading{Student Funding Information}
\begin{information}
    Councillors would like to see more communication and outreach for the 
    various sources of funding that students have available to hold events
    and improve student life. 

    Also councillors would like to see an investigation towards speeding up the
    process for a group to get money, and more awareness so that Feds clubs
    aren't always reaching out to external sources like Faculty Endowment 
    Funds. 

    The \vpi\ responded that they are currently discussing how to best
    advertise and market the Internal Funding Committee (IFC) and relevant
    funds. 
    
    Currently IFC meets bi-weekly, and proposals almost always came in on an
    off week.

    Most proposals came with less than 2 weeks notice, which is very difficult
    for students to work with as a time-frame. 

    In terms of the Student Life Endowment Fund, funding events doesn't come
    out of that fund, and it has historically been almost exclusively for
    investing in student lounges.

    A councillor also asked if it would make sense for Faculty Endowment Funds
    to sit together as a committee. Councillors agreed this would be great,
    but would be up to those organiations. 

    Councillors also expressed wanting to see more information on Faculty
    Funds on the website.
    
\end{information}

\heading{Cancelation of December Meeting}
\begin{information}
    \birt\ Council cancel its December 3rd meeting.
    \movers{\elizabeth}{\seneca}

    Councillors expressed that Council could vote on the Inigenous Student
    Policy over email if Feds needed to get that through early.

    A straw poll was held on whether to have an online vote.

    \carries.

    \jason, \alexander, \stephanie\ are noted against.
\end{information}
