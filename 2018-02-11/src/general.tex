\section*{Consent}

\heading{Approval of the minutes}
\begin{motion}
    \birt\ Council approve the minutes from January 14, 2017.
    \movers{\seneca}{\brian}

    \carries\ unanimously.
\end{motion}

\section*{Executive Reports}

Please see the attached written reports for the full reports from the Executive
to Student's Council. 

\heading{Report of the \pres}
\begin{information}

    The \pres\ has been busy working on mental health issues, the President's 
    Advisory Committee on Student Mental Health, and accessibility on campus.
    They have been addressing concerns that \nickta\ had on the possibility
    that the University of Waterloo had a mandatory leave policy for those
    diasnosed with mental illness; a policy that is currently implemented at
    the University of Toronto despite a ruling from the Huma Rights Commission
    that the practise was unconstitutional.  More broadly, the \pres\ has been
    investigating how the University and Feds can make education accessible and
    what is needed to acomodate people with physical and mental disabilities. 

    Regarding the President's Advisory Committee on Student Mental Health, the
    committee took the recomendations they have been collecting and are
    constructing real advice. \pres\ is working on two sections, and expects
    the final report will go to the President's office on Friday. 

    Additionally, the General Meeting Taskforce has met twice so far. They had
    a planning meeting, and are now moving into working groups related to each
    issue and possible solution. The \pres\ is expecting that they will present
    preliminary findings at the next General Meeting. 

    The Policy and Procedures Committee is still investigating ways to 
    expedite and improve the process of creating and implementing council 
    policies. The policy and research officer has drafted a report on best 
    practises from other unions.

    Finally, the \pres\ reported that international students will have a 
    predictable increases to tuition in the future. This year, it will be a 
    8.5\% increase for incoming students, and then 5\% per year after that for 
    current students. A councillor aked if increases affected enrolement, and
    the \pres\ responded that they did not.

\end{information}

\heading{Report of the \vpof}
\begin{information}

    The \vpof\ had a meeting with MathNEWs and MathSoc about two weeks ago,
    they came to a good resolution with the current Memorandum of 
    Understanding. They \vpof\ also had an individual meeting with MathNEWs 
    that was very positive. MathNEWs voiced concerns with financial
    oversight, and recognized their shared responsibility with Feds for 
    cooperating with audits and ensuringliability coverge. Going to continue to
    meet each month for updates including stepl to improve within Feds and 
    MathNEWs. They are working together to open channels for feedback and more 
    transparent comunication.

    The legal service survey will continue to be open for responses, and the
    \vpof\ asked Councillors to encourage their constituents to submit their
    opinion. They would like to recieve between 500 and 800 responses.
    Councillors can expect a report for the next council meeting, including
    whether there is sufficient interest in the proposed service. 

    Feds also held a job fair for Feds paid opportunities. About 175 students 
    visited with applications to apply for Feds part-time jobs. The event
    will continue to run at least twice a year, maybe three times. The \vpof\ 
    will continue to make sure hiring practises meet student's needs. 

    Email issues continue to affect grops across campus. The university 
    implemented a security feature that automatically marks forwarded emails 
    as spam, without informing or consulting key stakeholders. This means that
    internal emails from the ``@edu.uwaterloo.ca'' and ``@feds.ca add'' email
    domains are blocked.

    OHIP+ started in January, and is being marketed as ``free prescriptions'',
    but it only covers 4400 prescriptions that mainly target older populations.
    Since Feds Health and Dental Plan is an annual program, Feds had alreay
    bought in to another year due to a lack of information from the provincial
    government before the official release in Januay.

    The \vpof\ will be investigating how the changes will affect the current
    Health and Dental plan, the will make adjustments to the structure of the 
    plan as necessary. Councillors should keep in mind that not all students
    are eligable for OHIP+, and that the current Feds insurance is much more
    relevant to students than the current implementation of OHIP+. Feds will
    also be working to communicate the opt-out system that is a part of the
    plan. 

    A councillor commented that the student care opt-out system is circuitous
    and it should be a two-click process where students can just log in and 
    opt out. Councillors also expressed that the onus to provide proof of 
    supplementary coverage is often difficult to overcomem, although they
    recognized this is a necessary part of the process. 

    A councillor commented that they would like to see opinion editoials in
    Imprint from Fed's, MathNEWs', and MathSoc's perpesctive. The \vpof\ agreed
    to reach out to all parties with the suggestion, and exressed that they 
    personally have no issues with the suggestion.

\end{information}

\heading{Report of the \vpi}
\begin{information}

    The \vpi\ continues to meet with societies at the Committee of Presidents
    (COPs). Last meeting they passed a COPs procedure document to fomalize
    what COPs is and how it should operate. 

    They have also finalized all the events for this term, including a
    collaborative society event that will consist of kareoke in the Bomber on
    March 2, 2018.

    Societies also now have permission from IST to access the podiums in
    classrooms at no cost. 
    
    Finally, the society relations commissioner has compiled a goals document
    which is attached. 

    On the topic of wellness, ``Bell Lets Talk'' day happened, Feds partnered
    with athletics to spread awareness on campus with toques.  They are also
    working with health and wellness on an event that will help raise
    awarenss and reduce stigma associated with mental health through song, art,
    and other mediums. 

    The \vpi\ reported that Feds continues to unite campus on the crusade for
    wellness under one brand, to ensure that wellness initiatives have similar
    experiences and brandng.

    EOY has been rebranded in an effort to get more awareness and uptake. 

    Internal Administration Committee has met, the \vpi\ will distribute
    minutes for next meeting.

    The \vpi\ continues to meet with service advocacy directors. Working on
    Policy 42 review with them and the \pres.

    Provost's Advisory Committee on Equity (PACE) has held an open advisory 
    panel. They are deciding how to decide the working groups, incluing which
    issues will be prioritized. They are also determining how voting will work
    on the committee, collecting feedback from all relevent parties (faculty,
    staff, and students). They have received feedback on current policy and
    procedures, student employment rights, security for first year students, 
    and international issues. The \vpi\ encouraged councillors to reach out
    to her if they had any questions. The \vpi\ highlighted that we need more 
    student consultation groups, as students are the largest body on campus. 

    The \vpi\ reported that deadines had been set for the Student Life
    Endowment Fund (SLEF) and Volunteer Awards.

    After a consultation with students at the Cambridge campus, Feds will be
    prioritizing reducing bike theft in Cambridge and broadening the athletic
    opportunities available for the Campbridge campus. The Warrior Tribe is 
    also working on the Cambridge campus to help bridge the gap between the
    remote campus and the central one.  
    
\end{information}

\heading{Report of the \vpe}
\begin{information}

    Was not present. A councillor mentioned that EAC occurred, and that it was
    awesome. 

\end{information}

\section*{Regular}
\heading{Racialized Student Service}
\begin{motion}
    \birt\ Council approves the creation of a new student-run service
    catered to meet the needs of racialized students on campus, pending an
    approved fee increase of approximately \$1.20 at the March General Meeting.
    \movers{\antonio}{\brian}

    This campus has not been good at meeting the needs of
    racialized students on campus and the Equity and Diversity office has done
    poorly at meeting the needs of these students. \pres\ and \vpsl\ have been
    working with BASE and other equity groups on campus to try and meet these
    needs through Feds. 

    A councillor voiced support for the idea of improving our work on equity 
    on this campus, and in particular the idea of a racialized student 
    service. 

    \begin{motion}
        \birt\ Council ammend the motion to read:

        \begin{motion} 
            \birt\ Council approves-in-principles the creation of a new
            student-run service catered to meet the needs of racialized*
            students on campus, pending an approved fee increase of
            approximately \$1.20 at the March General Meeting and a service use
            creation proposal submitted to council for approval at the next
            regular meeting.
        \end{motion}
        \movers{\seneca}{\jennifer}

        Councillors raised concerns that the wording implies that council will
        later prevent the creation of the service despite the fee increase
        and approval-in-principle. 

        Councillors voiced support for Council's right to say no to proposals
        that are bad, or when it is not in the best interest of
        students-at-large.

        \begin{motion}
            \birt\ Council ammend the ammendment to add: 
            \begin{motion} 

                \bifrt\ Council recommends to the Board of Directors the
                inclusion of an agenda item for the General Meeting to approve
                a \$1.20 fee increase expressly contingent to Council’s
                approval of the creation of said service.

                \bifrt\ Council shall include as an agenda item for the March
                25th meeting to review the service use and creation proposal
                and executive cost assessment report (as presented to the
                General Meeting) for said service.

                \bifrt\ Council tasks the Campus Life Advisory Committee (CLAC)
                to review and recommend appropriate service execution of
                equity, diversity, and inclusivity services provided by the
                Federation of Students in their annual review of services and
                student life.

                \bifrt\ Council tasks the Executive Board with the exploration
                and negotiation of future funding arrangements for equity,
                diversity, and inclusivity services currently provided by the
                Federation of Students with the UWaterloo Equity Office in a
                manner similar to the joint resources of the UW Mates service
                cost-model.

            \end{motion}
            \movers{\senecca}{\brian}

            \carries. 
        \end{motion}

        The current ammendment is to change the motion to read: 
        \begin{motion} 
            \birt\ Council approves-in-principles the creation of a new
            student-run service catered to meet the needs of racialized*
            students on campus, pending an approved fee increase of
            approximately \$1.20 at the March General Meeting and a service use
            creation proposal submitted to council for approval at the next
            regular meeting.

            \bifrt\ Council recommends to the Board of Directors the inclusion
            of an agenda item for the General Meeting to approve a \$1.20 fee
            increase expressly contingent to Council’s approval of the creation
            of said service.

            \bifrt\ Council shall include as an agenda item for the March 25th
            meeting to review the service use and creation proposal and
            executive cost assessment report (as presented to the General
            Meeting) for said service.

            \bifrt\ Council tasks the Campus Life Advisory Committee (CLAC) to
            review and recommend appropriate service execution of equity,
            diversity, and inclusivity services provided by the Federation of
            Students in their annual review of services and student life.

            \bifrt\ Council tasks the Executive Board with the exploration and
            negotiation of future funding arrangements for equity, diversity,
            and inclusivity services currently provided by the Federation of
            Students with the UWaterloo Equity Office in a manner similar to
            the joint resources of the UW Mates service cost-model.
        \end{motion}

        \carries.
    \end{motion}

    The motion now reads: 
    \begin{motion} 
        \birt\ Council approves-in-principles the creation of a
        new student-run service catered to meet the needs of racialized*
        students on campus, pending an approved fee increase of
        approximately \$1.20 at the March General Meeting and a service use
        creation proposal submitted to council for approval at the next
        regular meeting.

        \bifrt\ Council recommends to the Board of Directors the inclusion
        of an agenda item for the General Meeting to approve a \$1.20 fee
        increase expressly contingent to Council’s approval of the creation
        of said service.

        \bifrt\ Council shall include as an agenda item for the March 25th
        meeting to review the service use and creation proposal and
        executive cost assessment report (as presented to the General
        Meeting) for said service.

        \bifrt\ Council tasks the Campus Life Advisory Committee (CLAC) to
        review and recommend appropriate service execution of equity,
        diversity, and inclusivity services provided by the Federation of
        Students in their annual review of services and student life.

        \bifrt\ Council tasks the Executive Board with the exploration and
        negotiation of future funding arrangements for equity, diversity,
        and inclusivity services currently provided by the Federation of
        Students with the UWaterloo Equity Office in a manner similar to
        the joint resources of the UW Mates service cost-model.
    \end{motion}

    \carries. \jason\ notably abstains. 

\end{motion}

\heading{MathNews/MathSoc Update}
\begin{motion}
    \birt\ Council remove this item from the agenda. 
    \movers{\seneca}{\brian}

    This item was addressed earlier.

    \carries\ unanimously.
\end{motion}

\heading{Various Motions by \nickta}
\begin{motion}
    \birt\ Council remove the following items proposed by \nickta\ from the 
    agenda due to her absense:

    \begin{itemize}
        \item Discussion on revalidating the student health and dental plan
        \item Discussion on ``University Mandated Leave of Absence Policy''
            that was recently withdrawn at the University of Toronto
        \item Discussion on introducing a mentorship program for new members of
            the FEDS council
    \end{itemize}

    \movers{\seneca}{\brian}
    \carries\ unanimously.
\end{motion}


\heading{Sustainability Policy Changes} 
\begin{motion}
    \birt\ Council amends Policy 25 (Sustainability) by adding the following
    provisions to the existing policy:
    \begin{motion}
        \whereas\ the University of Waterloo supports “practices and processes
        to reduce consumption of resources, minimize output of waste, and
        mitigate upstream and downstream environmental impacts from campus
        operations” (UW Policy 53, Environmental Sustainability, section 5.5),

        \whereas\ bottled beverages are the cause of large volumes of plastic
        waste; and

        \whereas\ the use of cheap reusable bottles filled with tap water is a
        low-cost, safe, and sustainable alternative;

        \bifrt\ the University of Waterloo, the Federation of Students, and the
        businesses that operate on campus should produce as little food waste
        as possible, including disposable containers, plates, cutlery, and
        bottles; and

        \bifrt\ the University of Waterloo should provide convenient and easy
        access to water fountains and water bottle refilling stations in all
        buildings.

    \end{motion}
    \movers{\jason}{\seneca}

    \begin{motion}

        \birt\ Council amend the motion to add the following clause:
        \begin{motion}
            \bifrt\ the Federation of Students and the University of Waterloo
            should operate on campus should educate the campus population about
            the benefits of sustainability and the environmental damages
            associated with food waste.
        \end{motion}
        and remove:
        \begin{motion}
            \whereas\ the University of Waterloo supports “practices and
            processes to reduce consumption of resources, minimize output of
            waste, and mitigate upstream and downstream environmental impacts
            from campus operations” (UW Policy 53, Environmental
            Sustainability, section 5.5),
        \end{motion}

        \carries\ as a friendly motion. 
    \end{motion}

    \carries.
\end{motion}

\heading{Refer Sustainability Policy to PPC} 
\begin{motion}
    \birt\ Council direct the sustainability policy to PPC to address potential 
    clerical issues and the operability of the policy. 
    \movers{\seneca}{\rebecca}
    \carries.
\end{motion}

\heading{Student Life Endowment Fund (SLEF) Appointment}
\begin{motion}
    \birt\ Council appoints \blank\ and \blank\ to the SLEF council seats, and
    \bifrt\ Council allows the executive team to appoint \blank\ and \blank\ to
    the SLEF at large seats.
    \movers{\jill}{\seneca}

    The \vpsl\ gave a brief overview of the committee. 

    \seneca\ nominates \stephanie\ to a Council seat. 
    \seneca\ nominates \elizebeth\ if \stephanie\ does not want the seat.
    \seneca\ nominates himself to a Council seat. 

    \begin{motion}
        \birt\ Council amend the motion to add ``in cosultation with the
        Speaker'' 

        The motion reads:
        \begin{motion}
            \birt\ Council appoints \blank\ and \blank\ to the SLEF council seats, and
            \bifrt\ Council allows the executive team in cosultation with the
            Speaker to appoint \blank\ and \blank\ to the SLEF at large seats.
        \end{motion}

        \carries\ as friendly.
    \end{motion}

    \carries\ unanimously.
\end{motion}

