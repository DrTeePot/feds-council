\section*{Consent}

\heading{Approval of the minutes}
\begin{motion}
    \birt\ Council approve the minutes from February xx, 2017.
    \movers{\seneca}{\jennifer}

    The discussion between \seneca\ and \brian\ on aproval by principle had
    the arguments made by each party swapped.

    \carries\ unanimously.
\end{motion}

\section*{Speakers Remarks}

The speaker informed the assembly that they would be enforcing speaking terms 
of a minute and a half, and that each councillor would have two speaking terms
on each discussion item. Interested members of the gallery may speak if a 
councillor yields them a speaking term.

\section*{Executive Reports}

Please see the attached written reports for the full reports from the Executive
to Student's Council. 

\heading{Report of the \pres}
\begin{information}

    The \pres\ thanked the members of the gallery for taking the time to attend 
    the meeting.

    They reported that the General Meeting, which occured on Wednesday March 
    21, 2018, had the second largest turnout in Feds history and was a very
    successful meeting. It was noted that the fee increase for a new racialized
    student service and a fee increase for service support staff both passed. 
    There is a motion on today's agenda to create the service with a 
    preliminary implementation plan.

    The \pacsmh\ released their final report, this is a very important document
    that will shape the approach the university and Feds takes to mental health
    advocacy on campus and within Feds for the next several years. The reveal
    event had over 500 people in attendance. The \pres\ noted that the report
    should address the points that students have been raising over the last few
    years, and that they felt the report was well-written and was backed by
    extensive consultation; over 300 people volunteered for the process. The 
    \pres\ encouraged students to speak with the executives if they had any 
    questions about the report or the process that was used to compile it.

    Regarding the recent Freedom of Expression policy that Council passed, the
    \pres\ has had many conversations with university administrators, senators,
    and the President of the University on the meaning of Freedom of Expression
    within the university and what the definition and limits of academic freedom
    are. The \pres\ is advancing the message that freedom of expression is not
    at odds with equity initiatives.

    A councillor inquired if Senate would be open to adopting the Chicago 
    principles; senators largely agree with the princples but there is no
    agreement on the tools that should be used to create a healthy and
    academically rigourous environment of debate. There was a town hall 
    to gather preliminary thoughts on definitions and how they would 
    affect faculty, staff, and students.

    The \pres\ also partnered with external student groups to work towards
    counting experiential learning opportunities towards the express entry
    process for international students. There has been good followup so far, 
    and Members of Parliment are expressing some support for the idea. They are
    planning a petition to have this considered for a formal response from the 
    House of Commons, which requires 500 respones. Feds will be targeting
    key times to generate engagement with the petition.

    A councillor inquired about the change to orientation fees that was
    approved at the General Meeting. Specifically, which body approves the 
    budget that is funded by these fees? The \pres\ will investigate this with
    their successor and the board.

    It was clarified that the definitions of academic freedom and freedom of 
    expressoin relate to the memorandum of agreement between the
    faculty and university which defines academic freedom and the rights and
    responsibilities that faculty are granted by it. The tools used to create
    a free and equitable environment of debate are important in order to get
    everyone on board, since faculty have large say through their memorandum
    of agreement.

\end{information}

\heading{Report of the \vpi}
\begin{information}

    The \vpi\ wrapped up \cops\ (COPs) meetings for the term, they had some key
    administartors in to talk to societies about the new grant that Waterloo 
    International is looking to grant. The committee members also had a 
    productive conversation about the role of societies in mental health on
    campus and how student leaders can work across faculties to be suppotive of
    students in all faculties. The student life department has also been 
    investigating the benefit of more partnerships between orientation and 
    societies, and having a greater society pressence on the Feds website.

    The \vpi\ also visited the pharmacy campus, and is investigating more
    opportunities to engage these students in May.

    They are working towards creating more centralized wellness initiatives on 
    campus.

    \iac\ finished their meetings for the term, and used some extra time to 
    discuss a higher level overview of the club system and its purpose. The 
    committee members dug deep into the club system, making sure the system is 
    the best it can be.

    Feds held Wrap-Up Week the prior week, which included massages, a snack
    cart, warrior wind-down, energy balls, and a variety of other activities
    and events.

    The \vpi\ alerted Council that Volunteer Appreciation would occur on 
    Monday, March 26, 2018. Councillors were invited to the event.

    The \vpi\ also reported that Cultural Caravan had the largest turnout in 
    Feds history, and that Leadership Awards had been chosen for the term.

    Finally, the \vpi\ has been working hard on transition, which has been 
    consuming large amounts of time.

    A councillor inquired if there would be any reports from the council 
    committees, including the attendance of those committees. They specifically
    were looking for attendance data to be attached to the minutes.

    A councillor asked if there was feedback from the society presidents on 
    possible ways for the \cops\ meeting to improve?

    A councillor asked if council could give recommendations for a possible
    candidate for the societies commisioner. \vpi\ replied that it may be a
    conflict of interest.

\end{information}

\heading{Report of the \vpof}
\begin{information}

    Not present.

    \seneca\ commmented that he had been working with the \vpof\ on a new
    budget procedure.

\end{information}

\heading{Report of the \vpe}
\begin{information}

    Not present.

\end{information}

\section*{Special Orders}
\heading{Health Promotion and Wellness at uWaterloo}
\begin{information}
    \pres\ introduced Jennifer McCorriston, Associate Director of Health
    Promotion within Campus Wellness; she has been key in the \pacsmh.

    She gave a presentation on wellness and the ways they are working to 
    unify health and wellness initiatives across campus in order to be
    preventative about mental health issues, instead of being reactive. Slides
    are attached.

    Councillors, societies, and audience members in the gallery are encouraged
    to contact the Feds executive, or Jennifer McCorriston directly, if they
    have questions or comments.

    It was noted that a survey will be sent to Council in the future to gather
    feedback on wellness spaces.

    \austin\ asked about how they envision convincing lecturers and 
    professors to adopt wellness strategies. During consultations, an academic
    panel that was called as part of the \pacsmh\ recommended policy
    changes that would affect classrooms. There will also be sessions on
    teaching and understanding students better. While those professors who
    attend sessions are likely those who least need these sessions, the 
    wellness is hopeful that there are other ways to further encourage 
    lecturers and professors to adopt wellness strategies in the classroom.

    A councillor asked why mental health and first aid training was not
    mandatory for professors and lecturers; specifically, what is the
    feasibility of requiring this training and why had it not been implemented.

    It was explained that there are challenges with implementing mandatory 
    training, one of which is academic freedom. Current academic freedom
    policies make it very difficult to mandate training for professors.

    Mandatory training that is mandated on campus is provincially mandated,
    so it was noted that it may be possible for Feds may get more traction
    at the provincial level if not from the administration. 
    
    \katie\ offered that the people in the faculty who most needed the
    training had the smallest amount of incentives to offer them.  The
    wellness team is also investigating incentives to encourage participation,
    and how to create greater buy in for training.  

    A councillor asked what resources were available for professors who seek 
    training. All training is available freely to any employee, including
    mental health first aid, QPR, physical first aid, etc. Sometimes these 
    training sessions are cancelled due to lack of enrollement. 

    A councillor commented on the need for coop advisors to be included in 
    the discussion and to have broad health and wellness training. 

    A councillor re-iterated that coop was a place that needed a lot of
    attention. Co-op advisors are a first point of contact for many students
    and they have no training and are often very callous. The councillor also 
    highlighted the inaccessible marketing of wellness services; many students
    do not know these services and iniatives exist, often due to cultural
    issues.

    A councillor asked whether there was consideration for offering services in
    students native languages. An anecdote about students who had switched into
    psychology just to provide these services for other students externally was
    added to the discussion. Inadequate understanding of cultural and
    socio-political backgrounds were highlighed as key barriers for students as
    well. 

    Many initiatives moving forward are in the \pacsmh\ Final Report, like
    doing more cultural sensitivity training. It was recognized that services 
    should be more cultural and language sensitive. A Mandarin speaking
    counsellor was recently hired to help with the language barrier that some
    students face in getting help. They are currently investigating what the
    health needs are for international students, what the service gaps are,
    and what should be done to support and understand these students.

    A councillor inquired if there was remote infrastructure for co-op
    students. There is some minimal councillors at satellite locations and
    they do their best to have remote calls and recommend global resourses.
    This is a service that is going to built on further with the \pacsmh. The
    councillor also recommended interfacing with Feds Coop Connection and Feds
    \icsn. 

    % TODO which commissioner? check attendance
    The commissioner encouraged councillors and students to talk to their 
    constituents, societies, and faculty about the issues and influence hiring
    policies.

    \begin{motion}
        \birt\ Councilors have extended speaking time to fully discuss the 
        issue at hand.
        \movers{\seneca}{\rebecca}

        The speaker extended the time.
    \end{motion}

    \cai\ expressed support for the report and the changing landscape, 
    including the role of adminsitrators and students. They highlighed that 
    there may be some professors who need more encouragement.

\end{information}

%TODO councillors get empathy training?

\section*{Regular}

\heading{Campus Dentist}
\begin{information}
    \seneca\ brought up a student concern with the Campus Dentist that is
    currently located in \slc. There have been reports that the business has
    lied to and taken advantage of students. A student was told they had 7
    cavities of dire quality, and 2 that needed to be filled immediately;
    however, when they got a second opinion from another provider and were told
    they had 2 cavities, neither of which were dire. The student also recieved
    x-rays from campus dentist and the second provider, and it was determined
    that the x-rays provided by Campus Dentist were not medically useful.

    Investigations showed that other students have reported similar issues with
    the Campus Dentist office on a variety of online forums.

    The councillor was wondering if there was any pressure Feds could apply to
    the dentistry through the lease agreement or the student care plan.

    A councillor expressed that students may associate the dentistry with Feds,
    which could pose a reputational risk.

    Executives will discuss this issue as a team as soon as possible, and will
    escalate to the board as necessary. The executive team encouraged the
    councillor to send any documentation on the issue to the executive team.

\end{information}

\heading{Publication of Council Records}
\begin{motion}
    \whereas\ the average academic tenure of a student is 5 years,
    \whereas\ old minutes are lacking in clarity,
    \birt\ Council requests the Board to investigate the feasibility of 
    removing minutes older than 5 years from the public website, and,
    \bifrt\ Board enact any resolution on Council's behalf to this effect for
    ratification by the next Council meeting.
    \bifrt\ minutes shall still be available for access by members.
    \movers{\nickta}{\stephenie}

    This is due to inaccuracies in old minutes, potential confusion, and the
    lack of workmanship in older documents.

    A councillor recommended that it be 7 years to be in-line with other
    governent audit requirements.

    \carries\ unanimously. 

\end{motion}

\heading{Agenda}
\begin{motion}
    \birt\ Council flip the last two agenda items
    \movers{\seneca}{\rebecca}
    \carries\ unanimously.
\end{motion}

\heading{Racialized Student Service}
\begin{motion}
    \birt\ that Council approves the creation of an equity student-run service
    looking to meet the unmet needs of racialized students and address racism
    on campus.
    \movers{\antonio}{\rebecca}

    \hyperref[equity]{The full proposal is attached.}

    A councillor expressed dissapointment that the full proposal had not been
    seen by students voting at the General Meeting, had not been seen fully by
    council, and had found research that showed some of the measures the 
    service looked to implement may decrease diversity.

    A councillor clarified that the General Meeting would never have seen this
    proposal in any process, as the aproval process is delegated to Council and
    the Campus Life Advisory Committee. 

    The \pres\ responded that themselves and the \vpi\ had been working hard
    to open up the consultation process to as many students as possible, and
    the prosposal is grounded in research that was done by the government and
    in particular related to the post-secondary sector. This proposal only 
    includes a proposal and a rough implementation plan, implementation 
    planning will happen over the next two terms.

    A councillor asked about the lack of awareness in the student body, and 
    the ways that students were consulted. 

    \begin{motion}
        \birt\ Council defer the motion to a referendum.
        \movers{\harsh}{\alexander}

        \seneca\ and \antonio\ motion to rule this motion out of order.

        \seneca\ spoke that Council has the full authority to determine
        services that are created and removed, and the way these services make
        it into the university.

        \harsh\ said that referendum should be used to determine student
        opinion.

        \elizabeth\ ruled the motion out of order, as a motion to defer is not
        a valid motion that Council can enact.
    \end{motion} 

    A councillor expressed that the \clac\ was involved and would continue to
    be involved in creating the service. The service would not be created 
    behind closed doors. 

    The remainder of their time was yielded to \elisa\ in the gallery, who
    thanked council and the executives for hearing the concerns of students
    and consulting them in the process. 

    It was emphasized by members of the gallery that an incredible amount of
    consultation that occured; a large number of students were consulted and
    voted in favour of the service, and students believe this service is needed.

    \amanda\ expressed the importance of services that are available to 
    advocate for marginalized students on campus and the ability of students
    to express feedback to the university as an representative group. These
    services allow students to access important administrators in the 
    university, and have greater legitimacy within the university. Services
    also provide funds for events and promotion of the issues that these 
    students face, including advocacy campaigns. Finally, they expressed how 
    having a space creates a sense of community. They emphasized that it was
    shocking that Feds was only considering a racalized student service now.

    A gallery member expressed support for a referendum.

    A councillor expressed support for the motion, and apologized that the
    service had not been created sooner. 

    A gallery member commented on the disapointment felt by international 
    students. They echoed that the lack of support for racialized students at
    this university is crazy, especially given the high rates of hate crime in
    Waterloo.

    A councillor suggested making the fee opt-out. 

    A gallery member expressed support for fees, paying collectively agreed
    upon fees is an important part of existing as a student body.

    A councillor highlighed the oversight that is present in the proposal, 
    and expressed support.

    A councillor expressed that there was a certain amount of due dillegence 
    that occurs on campus, and in the creation of this proposal. 

    On the topic of councillors reprenting students and the possible arguments
    for a referendum: councillors expressed that they did represent students
    and that there was a process that could be followed if constituents or
    councillors felt they were not representing students. Councillors
    commented on the work involved in the Councillor position, and encouraged
    any member of Council who did not feel like they could represent their
    constituents to do better at reaching out and interacting with their 
    constituents.

    A gallery member asked what the implication of the motion passing
    and failing had.

    A councillor responded that the motion approved the proposal and
    implementation of a service beyond the in-principle motion that was
    approved at the last meeting. This would then get passed off to the \vpi\
    to create a full service structure and the service manager and \clac\ to
    implement the full service.

    A gallery member commented that this was discussed on the Reddit, and there
    was very little engagement.

    A motion was made to postpone the vote:

    \begin{motion}
        \whereas\ the proposal for a Racialized Student Service was rushed and
        few students were consulted,

        \whereas\ racism is a real issue on campus for which a comprehensive
        solution is needed,

        \birt\ Feds postpone the vote on a Racialized Student Service to the
        July council meeting

        \bifrt\ Feds consult with the general student body on the best solution
        to deal with incidents of racism on campus, especially with
        international students and with students effected by racism.

        \bifrt\ Campus Life Advisory Committee will compile the results of the
        consultation and work on editing and finalizing a proposal to deal with
        racism, whether that be a Feds Service or a different solution.

        \bifrt\ should council decide that the proposal for a Racialized
        Student Service be the best solution to deal with racism, a Winter 2019
        implementation deadline be kept.
        \movers{\alexander}{\seneca}

        The mover spoke to the motion, saying that international students
        should be more involved and that the service should have the same
        implementation date. They emphasized that students deserve a more
        thoughtful implementation.

        The \pres\ spoke against the motion, stating that consultation did
        occur and the arbitrary line of consultation for this service in
        particular seems disingenuous.

        A councillor inquired what the precise timeline was for the 
        implementation plan. The \vpi\ is looking to have coordinators hired
        and started in Spring 2018 to begin impelementation and further
        consultation, and again through Fall 2018. The full service will launch
        in Winter 2019.

        A councillor expressed that the whereas statements were disingenous,
        and asked that the motion be changed to
        \begin{motion}
            \whereas\ the proposal for a Racialized Student Service was felt to
            be rushed by some members of the Federation,

            \whereas\ racism is a real issue on campus for which a comprehensive
            solution is needed,

            \birt\ Feds postpone the vote on a Racialized Student Service to the
            July council meeting

            \bifrt\ Feds consult with the general student body on the best solution
            to deal with incidents of racism on campus, especially with
            international students and with students effected by racism.

            \bifrt\ Campus Life Advisory Committee will compile the results of the
            consultation and work on editing and finalizing a proposal to deal with
            racism, whether that be a Feds Service or a different solution.

            \bifrt\ should council decide that the proposal for a Racialized
            Student Service be the best solution to deal with racism, a Winter 2019
            implementation deadline be kept.

            \movers{\stephanie}{\alexander}

            Ammendment is friendly.
        \end{motion}

        A gallery member asked what threshold of oppression is required to
        create a service.

        A councillor expressed that they did not agree with the ammendment due
        to the decrease in consultation that it would create.

        The motion fails. \katie, \jennifer, \seneca, \nickta, \rebecca, \cai,
        and \stephanie\ were noted against.

    \end{motion}

    The motion remains unchanged.

    A councillor expressed a number of changes they wished to see in the
    proposal, but presented no ammendments.

    \begin{motion}
        \birt\ Council increase the speaking time of each councillor to 5
        minutes.
        \movers{\seneca}{\cai}
        \carries\ unanimously.
    \end{motion}

    Councillors were given increased speaking time.

    A councillor moved to change the voting method:

    \begin{motion}
        \birt\ Council conduct the vote on this motion by roll call vote.
        \movers{\harsh}{\nickta}

        Councillors expressed the benefits of having votes be private,
        including the freedom from social pressure. Councillors also expressed
        some down sides, including the potential for a lack of accountability.

        A councillor asked the mover to speak to their motion. The mover simply
        that the record would be useful.

        A vote was conducted and the motion fails.
    \end{motion}

    The motion remains unchanged. Voting will occur as usual.

    \begin{motion}
        \birt\ Council approves the creation of a service as outlined in the
        Service Creation and Cost-Benefit Proposal contingent on:

        \begin{itemize}
            \item \clac\ oversight through regular reporting of the VP Student
                Life (formerly VP Internal), Services Manager, and relevant
                coordinators as necessary, and 

            \item Operational oversight of the implementation of the
                student-run service by \clac\ including, but not limited to,
                the development of metrics of success and an outline of the
                deliverables the service shall be assigned, and

            \item \clac\ developing procedural restrictions on political
                activism of Federation services, and

            \item The Executive Board securing resource sharing
                arrangements of with relevant Universities bodies
                including, but not limited to the UW Office of Human
                Rights, Equity and Inclusion;
        \end{itemize}
        \movers{\seneca}{\alexander}

        The mover expressed that the motion was a great idea, and the service
        was needed. The ammendment aims to add more oversight and cost sharing
        to reduce costs for students and make sure the service is serving the
        demographic.

        A councillor expressed support for the ammendment.

        A councillor expressed confusion over the third item, specifically 
        regarding the definition of ``political activism'' and a ``political
        stance''.

        \katie\ pointed out that the third point was not in scope of this
        motion, and that the fourth point not restrict the creation of the
        service on University buy-in.

        The mover modified their motion to:

        \begin{motion}
            \birt\ Council approves the creation of a service as the Service Creation
            and Cost-Benefit Proposal contingent on:

            \begin{itemize}
                \item CLAC oversight through regular reporting of the VP Student
                    Life (formerly VP Internal), Services Manager, and relevant
                    coordinators as necessary, and 

                \item Operational oversight of the implementation of the
                    student-run service by CLAC including, but not limited to, the
                    development of metrics of success and an outline of the
                    deliverables the service shall be assigned, and

                \item The exploration and negotiation of future resource sharing
                    arrangements currently provided by the Federation of Students with
                    the relevant University bodies including, but not limited to the UW
                    Office of Human Rights, Equity and Inclusion;

            \end{itemize}

            \carries\ as a friendly ammendment.
        \end{motion}

        \begin{motion}
            \birt\ Council split the second item as its own motion.

            \carries\ as a friendly ammendment.
        \end{motion}

        Council begins a vote on the current motion, which reads:
        \begin{motion}
            \birt\ Council approves the creation of a service as the Service Creation
            and Cost-Benefit Proposal contingent on:

            \begin{itemize}
                \item CLAC oversight through regular reporting of the VP Student
                    Life (formerly VP Internal), Services Manager, and relevant
                    coordinators as necessary, and 

                \item The exploration and negotiation of future resource sharing
                    arrangements currently provided by the Federation of Students with
                    the relevant University bodies including, but not limited to the UW
                    Office of Human Rights, Equity and Inclusion;

            \end{itemize}
        \end{motion}

        \carries\ unanimously.
    \end{motion}

    Council begins to consider the second motion, which reads:
    \begin{motion}
        \birt\ Council approves the creation of a service as the Service Creation
        and Cost-Benefit Proposal contingent on:
        \begin{itemize}
            \item Operational oversight of the implementation of the
                student-run service by CLAC including, but not limited to, the
                development of metrics of success and an outline of the
                deliverables the service shall be assigned, and
        \end{itemize}

        After some discussion, a motion is presented to postpone the item to
        new business.

        \begin{motion}
            \birt\ Council postpone consideration of operational oversight for
            services to new business. 
            \movers{\seneca}{\katie}

            \carries\ unanimously.
        \end{motion}

        The motion is postponed till new business.
    \end{motion}

    The motion now reads:

    \begin{motion}
        \birt\ Council approves the creation of a service as the Service
        Creation and Cost-Benefit Proposal contingent on:

        \begin{itemize}
            \item CLAC oversight through regular reporting of the VP Student
                Life (formerly VP Internal), Services Manager, and relevant
                coordinators as necessary, and 

            \item The exploration and negotiation of future resource sharing
                arrangements currently provided by the Federation of Students
                with the relevant University bodies including, but not limited
                to the UW Office of Human Rights, Equity and Inclusion;
        \end{itemize}
    \end{motion}

    \begin{motion}
        \birt\ Council adds the word ``optional'', such that the service 
        ``advocates optional anti-racism, anti-opression, \ldots'' 
        \movers{\alexander}{\seneca}

        \carries\ as friendly.
    \end{motion}

    \begin{motion}
        \birt\ Council removes ``For example, Ryerson Students' Union funds and
        operates a student-run service called ``Racialised Students'
        Collective'' that works with Ryerson's community to eliminate racism
        and xenophobia both on and off campus through education and advocacy
        initiatives''.
        \movers{\alexander}{\seneca}

        The motion was withdrawn after brief discussion.
    \end{motion}

    \carries. \harsh\ noted against. \nickta, and \alexander\ noted for.

\end{motion}
    
\heading{Feds Mascot}
\begin{motion}
    \whereas\ mascots are the embodiment of the collective identity and
    brand of an institution, and

    \whereas\ there is significant student desire for our official mascot
    and branding to incorporate a goose, and

    \whereas\ the Waterloo community at large has come to view geese as an
    unofficial symbol of the University of Waterloo, including through
    snapchat filtering and the sale of plush goose stuffed animals, now
    therefore,

    \birt\ the mascot of the Federation of Students shall be a Goose; and

    \bifrt\ the Council shall create an ad hoc “WaterFowl Mascot
    Committee”, to be chaired by the President or a representative
    thereof, which shall during a one (1) year period determine naming,
    design, and style of the Feds Mascot through student submission; and

    \bifrt\ this committee shall hold a survey of students for the naming
    of said goose, from which Council shall choose a name of the top
    contenders; and

    \bifrt\ Council tasks the Executive to work with the University to make
    a goose the official university mascot. 

    \bifrt\ The mascot of the Federation of Students shall not be contrued as
    official branding or marketing of the Federation.

    \movers{\seneca}{\jennifer}

    Some councillors expressed support for recognizing the goose as a signifier
    of the community.

    There was a motion presented to make this just for the current term, but
    was ruled out of order. 
    
    \carries\ unanimously. 
\end{motion}

\section*{New Business}

\heading{Cancel April Meeting}
\begin{motion}
    \birt\ Council cancels its next meeting.

    Debate occured around whether Council should meet during or before the exam
    period, if a meeting is necessary. It was determined that a meeting was
    likely not necessary, and in the event that a an urgent item needed 
    Council approval an email vote would occur.

    Council expressed unanimous support, so the Speaker will not call a meeting
    in April.
\end{motion}

\heading{Partisan Activism in Services}
\begin{motion}
    \birt\ Council tasks \ppc\ to develop procedural restrictions on partisan
    activism within Federation of Student's services.
    \movers{\seneca}{\katie}

    \carries\ unanimously.
\end{motion}

\heading{Operational Oversight of Services}
\begin{motion}
    \birt\ Operational oversight of the implementation of the student-run
    services by \clac\ including, but not limited to, the development of metrics
    of success and an outline of the deliverables the services shall be
    undertaken and \ppc\ shall develop procedures that reflect this.
    \movers{\seneca}{\katie}

    This motion was originally made earlier in the meeting.

    \carries\ unanimously.
\end{motion}
