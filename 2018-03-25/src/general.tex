\section*{Consent}

\heading{Approval of the minutes}
\begin{motion}
    \birt\ Council approve the minutes from February xx, 2017.
    \movers{\seneca}{\jennifer}

    The discussion between \seneca\ and \brian\ on aproval by principle that were 
    swapped.

    Board approves the minutes in one of the sections.

    \carries\ unanimously.
\end{motion}

\section*{Speakers Remarks}

The speaker informed the body of the enforcement of speaking terms, a minute
and a half and two terms per councllor. Gallery members may speak if a
councillor yields them a speaking term.

\section*{Executive Reports}

Please see the attached written reports for the full reports from the Executive
to Student's Council. 

\heading{Report of the \pres}
\begin{information}

    Thanks for coming out everyone

    GM wentwell, it wason wednesday. second largest turnout in feds history.

    Of note is the fee increase passed for the racialized stunet service and
    for the staff member to expand the service. A motion is on the agenda today
    to create the service.

    PASMH released their report, a vry important document that will shape the mental
    health advoccy on campus and within feds for the next several years. The
    reveal event had upwards of 500 people in attendance. Should address
    the points that students have been raising over the last few years, and
    the report was very well done and had a lot of consultation. OVer 300
    people volunteered for the process. If students have any questions
    pleae speak to the executives and \pres\ would be happy to help with that.

    Freedom of expression, council passed this, numerous conversations with
    upper admin and senators and the president on what freedome of
    expressino is and what academic freedom is. We need to advance the message
    that freedom of expression is at odds with equity. How do we do this without
    seeming like we are getting "ahead ofthe curve" wrt whats happening across
    the street. 

    A councillor asked if senate would be open to adopting the chicago principles,
    senators largely like the princples but there is no agreement on the tools 
    that should be used.

    A town hall occured to try and pen down some definitions ad how they would 
    affect faculty staff and students

    partnered with student groups across the xxx to make sure that experiential
    learning opportunities count towards the express entry process for 
    international students. Currently good followup and Mp's are expressing
    some support. only need 500 respones on the house of commons petition to 
    get this on the agenda for a formal response. Will be targetting key times
    to get those.

    A councillor inquired about the orientatino fees. Will that remain under
    fees, who approves the budget that is presented relatig ot these fees? \pres\ 
    will investigate with their successor and board.

    The definitions of academic freedom and FoE relate to the MOA between the
    faculty and university which defines academic freedom. The tools are 
    important in order to get everyone on board, since faculty have large
    say through their MOA.

    There have been no updates from November and March. Council will continue
    to see get a report on this.

    A councillor asked if there was more information on the event and initiative
    to counct exp ed as experience for international students.

\end{information}

\heading{Report of the \vpi}
\begin{information}
    Wrapped up COPS meeting for the term, had some key administartors in to
    talk about the new grant that waterloo international is looking to give
    towards all members of the uni campus.

    HAd a good conversation with societies about what their role is in mental
    health on capus and how we can work across faculties to be suppotive of
    everyone.

    More partnerships between orientation and societies, and more pressense
    on the fed website.

    Went out to pharmacy, looking for more opportunities to engage these students
    come may.

    More centralized wellness initiatives

    IAC wrapped up their last meeting. Something they had time for was to talk
    about higher level overview of the club system adn it's purpose. Into the
    weeds of the club system, making sure it's the best it can be.

    Wrap up week was this week, massages, snck cart, warrior wind down, energy
    balls, variety of to others.

    volunteer appreciation is tmo, councillors are invited.

    Cultural caravan hd the largest turnout in feds history

    leader awards were chosen this morning

    Transition takes a large amount of time

    A councillor inquired if there would be any reports from the council 
    committees, including the attendance of those committees. 

    Specifically wants the attendance data (to be attached to minutes).

    Notes from meetings fr the past year?

    Is there feedback frm the presidents on how cops can improve?

    Can council give recommendatinos for societies commisioners? \vpi\ replied
    that it may be a conflict of interest.

\end{information}

\heading{Report of the \vpof}
\begin{information}

    Not present.

    \seneca\ commmented that he had been working with the \vpof\ on a new
    budget procedure.

\end{information}

\heading{Report of the \vpe}
\begin{information}

    Not present.

\end{information}

\section*{Special Orders}
\heading{Health promotion and wellness at uwaterloo}


Antonio gave an introduction, Jennisfer has been key in the PAC SMH.
Jennifer mccorriston
associate director, health promotion, 
campus wellness

Slides are attached.

Councillors, societies, and audience members are encouraged to reach out if
they have questions or comments.

A survey will be sent to Council to gather feedvcak on wellness spaces

Science pres person asked about how they envision convincing lecturers and 
professors to adopt wellness strategies. 

There was an academic panel that met as part of the PAC SMH tht recommended
some policy changes thatwould affect classrooms. There will also be sessions on
teaching and understanding students better. While those professors who attend
sessions are likely the choir, there are likey ways to further encourage this.

A councillor asked why it wasn't mandatory for mental health and first aid
training to be mandatory for professors. What is the feasibility of doing
this and why hasn't it been implemented.

There are challenges with implementing this, one of which is academic freedom,
where the academic freedom policies make it very difficult to mandate training.

Mandatory training that is mandated on campus is provincially mandated,
possible to push this from the province if not the administration.

A councillor asked what resources are available for professors to gain
training? All training is available freely to any employee, including mental 
health first aid, QPR, physical first aid, etc. Sometimes training is cancelled
due to lack of enrollement. 

Often people need incentives to do things, also looking at how to create buy
in for training.

A councillor was present at a roundtable with other campus partners and an MP,
asked whether there was consideration for offering services in students native
languages. Some students had switched into psychology just to provide these
services for other students. Cultural and socio-political backgrounds are 
important as well. 

A number of those things, like doing more cultural sensitivity training, is in
the report. Making the services more culturally and language sensitive. Mandarin
speaking counsellor was recently hired. Currently investigating what the health
needs are for international students and what service is needed, what acn we 
do to support and understand these students.

A councillor inquired if there was remote infrastructure for coop students. 
There is some minimal councillors at satellite locations and they do their
best to have remote calls and recommend global resourses. This is osmething
that is going to built on further with the PAC SMH. The councillor also 
recommended interfacing with FEds Coop Connection and FEds ICSN. 

The engineering B president offered that the people in the faculty who most
needed the training had the smallest amount of incentives to offer them. 

The commissioner encouraged councillors and students to talk to their 
constituents, societies, and faculty about the issues and influence hiring
policies.

\seneca\ and \rebecca\ moved to extend the speaking time. The speaker extended 
the time.

A councillor commented on the need for coop advisors to be included and have
this training. 

A councillor re-iterated that coop was a place that needed a lot of work. 
Coop advisors are a first point of contact for many students and they have
no training and are often very callous. They also talked bout the ways that
the marketing is not accessible, and many students do not know these services
and iniatives exist, often due to cultural issues.

The Environment president expressed support for the report and the changing
landscape, ad the role of adminsitrators and students; there may be some 
professors who need more encouragement.

%TODO councillors get empathy training?

\section*{Regular}

\heading{Campus Dentist}
\begin{information}
    \seneca\ brought up student concern with the Campus Dentist that is currently
    located in SLC. There is reports that the business has lied to students and
    taken advantage of them. A student was told they had 7 caveties of dire
    quality, and 2 that needed to be filled immediately. They got a second opinion
    from another provider and were told they had 2 caveties, neither of which
    were dire. Complementary xrays from campus dentist and the second provider.
    campus dentst xrays were apparently not medically useful.

    The councillor was wondering if there was pressure we could apply to the
    dentistry through the lease or the student care plan.

    A councillor expressed that students may associate the dentistry with Feds,
    which could be a reputatinal hazard.

    Executives will talk about this issue as a team as soon as possible, and
    will escalate to board as necessary. They encouraged him to send the formal
    document he was preparing to them.

\end{information}

\heading{Removing old minutes from Feds website}
\begin{motion}
    Feds remove minutes older than 5 years from the website. This is due
    to inaccuracies in old minutes, potential confusion, and the lack of 
    workmanship in documents.

    A councillor recommended that it be 7 years to be in line with other
    governent audit requirements.

    \whereas\ the average academic tenure of a student is 5 years,
    \whereas\ old minutes are lacking in clarity,
    \birt\ Council requests the Board to investigate the feasibility of 
    removing minutes older than 5 years from the public website, and,
    \bifrt\ Board enact any resolution on Council's behalf to this effect for
    ratification by the next Council meeting.
    \bifrt\ minutes shall still be available for access by members.
    \movers{\seneca}{\stephenie}

    \carries\ unanimously. 

\end{motion}

\heading{Agenda}
\begin{motion}
    \birt\ Council flip the last two agenda items
    \movers{seneca}{\rebecca}
    \carries\ unanimously.
\end{motion}

\heading{ }
\begin{motion}
    \birt\ that Council approves the creation of an equity student-run service
    looking to meet the unmet needs of racialized students and address racism
    on campus.
    \movers{\antonio}{\rebecca}

    The full proposal is attached.

    A councillor expressed dissapointment that the full proposal had not been
    see by students voting at the General Meeting, had not been seen fully by
    council, and had found research that showed some of the measures the 
    service looked to implement may decrease diversity.

    A councillor clarified that the General Meeting would never have seen this
    proposal in any process, as the aproval process is delegated to Council and
    the Campus Life Advisory Committee. 

    The \pres\ responded that themselves and the \vpi\ had been working hard
    to open up the consultation process to as many students as possible, and
    the prosposal is grounded in research that was done by the government and
    in particular relates to the post-secondary sector. This proposal only 
    includes a proposal and a rough implementation plan, implementation 
    planning will happen over the next term.

    A councillor asked about the lack of awareness in the student body, and 
    the ways that 

    \begin{motion}
        \birt\ Move this to referendum
        \movers{\harsh}{\alexander}

        \seneca\ and \antonio\ motion to rule this motion out of order.

        \seneca\ spoke that Council has the full authority to determine services that are created
        and removed, and the way these services make it into the university.

        \harsh\ said that referendum should be used to determine student opinion.

        \elizabeth\ ruled the motion out of order.
    \end{motion} 

    A councillor expressed that the \clac\ was involved and would continue to
    be involved in creating the service. The service would not be created 
    behind closed doors. 

    Th remainder of their time was yielded to Alyssa in 
    the gallery, who thanked council and the exectutives for heaing them and
    consulting them in the process. 

    There was a lot of consultation that occured, and a large number of students
    who were consulted and voted in favour of a service that they felt was 
    needed.

    Glwo coord expressed that a service that is availabe to advocate for you and
    the ability to express feedback. And gain access to hire ups in the uni.
    Services have power within the uni, and money to have events and promotion.
    Advocacy campaigns. Space creates a sense of community. It is shocking that
    we only have a racalized student servie now.

    Another galery member expressed support for a referendum.

    A councillor expressed support for the motion, and apologized that the
    service was not created sooner. 

    A gallery member commented on the dissapointemnt felt by international 
    students that this university does not have this service is crazy,
    especailly given the high rate of hate crime in waterloo and the fact that
    surrounding services.

    A councillor suggested making the fee opt-out. 

    A gallery member expressed support for fees and that it was an importnat
    part of agreeing as a student body.

    A councillor highlighed the oversight that is present in the proposal, 
    and expressed support.

    A councillor expressed that there was a certain amount of due dillegence 
    that occurs on campus, and in the creation of this proposal. 

    On the topic of councillors reprenting students; councillors expressed that
    they did represent students and that there was a process that could be
    followed if constituents or councillors felt they were not representing
    students.

    A gallery member asked what the implication of the motion passing
    and failing had.

    A councillor responded that the motion approved the proposal and
    implementation of a service beyond the in principle motion. This would then
    get passed off to the \vpi\ to create a full service structure and the
    service manager and \clac\ to create.

    A gallery member commented that this was discussed on the Reddit, adn there
    was very little engagement.

    \begin{motion}
        \whereas\ the proposal for a Racialized Student Service was rushed and
        few students were consulted,

        \whereas\ racism is a real issue on campus for which a comprehensive
        solution is needed,

        \birt\ Feds postpone the vote on a Racialized Student Service to the
        July council meeting

        \bifrt\ Feds consult with the general student body on the best solution
        to deal with incidents of racism on campus, especially with
        international students and with students effected by racism.

        \bifrt\ Campus Life Advisory Committee will compile the results of the
        consultation and work on editing and finalizing a proposal to deal with
        racism, whether that be a Feds Service or a different solution.

        \bifrt\ should council decide that the proposal for a Racialized
        Student Service be the best solution to deal with racism, a Winter 2019
        implementation deadline be kept.

        \movers{\alexander}{\harsh}


        The mover spoke to the motion, saying that international students
        should be more involved and that the service should have the same
        implementation date. The students deserve a more thoughtful 
        implementation.

        The \pres\ spoke against the motion, saying that consultation did occur
        and the arbitrary line of consultation for this service in particular
        seems disingenuous.

        A councillor inquired what the timeline was.

        The \vpi\ is looking to have coordinators hired and started in Spring 
        2018 to begin impelementation and consultation, and again through
        Fall 2018. The full service will launch in Winter 2019.

        A councillor expressed that the whereas statements were disingenous,
        and asked that the motion be changed to
        \begin{motion}
            \whereas\ the proposal for a Racialized Student Service was felt to
            be rushed by some members of the Federation,

            \whereas\ racism is a real issue on campus for which a comprehensive
            solution is needed,

            \birt\ Feds postpone the vote on a Racialized Student Service to the
            July council meeting

            \bifrt\ Feds consult with the general student body on the best solution
            to deal with incidents of racism on campus, especially with
            international students and with students effected by racism.

            \bifrt\ Campus Life Advisory Committee will compile the results of the
            consultation and work on editing and finalizing a proposal to deal with
            racism, whether that be a Feds Service or a different solution.

            \bifrt\ should council decide that the proposal for a Racialized
            Student Service be the best solution to deal with racism, a Winter 2019
            implementation deadline be kept.

            \movers{\stephanie}{\alexander}
        \end{motion}

        Ammendment is friendly.

        A gallery member asked what the threashold of oppression is required 
        to create a service.

        A councillor expressed that they did not agree with the ammendment due
        to the decrease in consultation that it would create.

        The motion fails. \katie, \jennifer, \seneca, \nickta, \rebecca, \cai,
        and \stephanie\ were noted against.

    \end{motion}

    The motion remains unchanged.

    A councillor expressed a number of changs they wished to see in the
    propsoal.

    \begin{motion}
        \birt\ Council increase the speaking time of each councillor to 5
        minutes.
        \movers{\seneca}{\cai}
        \carries\ unanimously.
    \end{motion}

    \begin{motion}
        \birt\ Council conduct the vote on this motion by roll call vote.
        \movers{\harsh}{\nickta}

        Councillors expressed the benefits of having votes be private, inclding
        the freedom from social pressure. Councillors also expressed some down
        sides, including the 

        A councillor asked the mover to speak to their motion. The mover simply
        that the record would be useful.

        Motion fails.
    \end{motion}

    The motion remains unchanged. Voting will happen as usual.

    \begin{motion}
        \birt\ Council approves the creation of a service as the Service Creation
        and Cost-Benefit Proposal contingent on:

        \begin{itemize}
            \item CLAC oversight through regular reporting of the VP Student
                Life (formerly VP Internal), Services Manager, and relevant
                coordinators as necessary, and 

            \item Operational oversight of the implementation of the
                student-run service by CLAC including, but not limited to, the
                development of metrics of success and an outline of the
                deliverables the service shall be assigned, and

            \item CLAC developing procedural restrictions on political
                activism of Federation services, and

            \item The Executive Board securing resource sharing
                arrangements of with relevant Universities bodies
                including, but not limited to the UW Office of Human
                Rights, Equity and Inclusion;
        \end{itemize}
        \movers{\seneca}{\alexander}

        The mover expressed that the motion was a great idea, and the service was
        needed. The ammendment aims to add more oversight and cost sharing to 
        reduce costs for students and make sure the service is serving the 
        demographic.

        A councillor expressed support for the ammendment.

        A councillor expressed confusion over the third item. What is political
        activism? What counts as a political stance?

        \katie\ expressed that the third point was not in scope of this motion,
        and that they would rather have the fourth point not show 

        The mover modified their motion to:

        \begin{motion}
            \birt\ Council approves the creation of a service as the Service Creation
            and Cost-Benefit Proposal contingent on:

            \begin{itemize}
                \item CLAC oversight through regular reporting of the VP Student
                    Life (formerly VP Internal), Services Manager, and relevant
                    coordinators as necessary, and 

                \item Operational oversight of the implementation of the
                    student-run service by CLAC including, but not limited to, the
                    development of metrics of success and an outline of the
                    deliverables the service shall be assigned, and

                \item The exploration and negotiation of future resource sharing
                    arrangements currently provided by the Federation of Students with
                    the relevant University bodies including, but not limited to the UW
                    Office of Human Rights, Equity and Inclusion;

            \end{itemize}
        \end{motion}

        \begin{motion}
            \birt\ Council split the second item as its own motion.
            \movers{\seneca}{}

            \carries\ as friendly.
        \end{motion}

        Council begins a vote on the first motion, which reads:
        \begin{motion}
            \birt\ Council approves the creation of a service as the Service Creation
            and Cost-Benefit Proposal contingent on:

            \begin{itemize}
                \item CLAC oversight through regular reporting of the VP Student
                    Life (formerly VP Internal), Services Manager, and relevant
                    coordinators as necessary, and 

                \item The exploration and negotiation of future resource sharing
                    arrangements currently provided by the Federation of Students with
                    the relevant University bodies including, but not limited to the UW
                    Office of Human Rights, Equity and Inclusion;

            \end{itemize}
        \end{motion}

        \carries\ unanimously.

        Council begins to consider the second motion, which reads:
        \begin{motion}
            \birt\ Council approves the creation of a service as the Service Creation
            and Cost-Benefit Proposal contingent on:
            \begin{itemize}
                \item Operational oversight of the implementation of the
                    student-run service by CLAC including, but not limited to, the
                    development of metrics of success and an outline of the
                    deliverables the service shall be assigned, and
            \end{itemize}
        \end{motion}
        \seneca\ moves to postpone consideration of operational oversight for
        services to new business. Seconded by \katie. 

        \carries\ unanimously.

    \end{motion}

    The motion now reads:
    \begin{motion}

        \birt\ that Council approves the creation of an equity student-run
        service looking to meet the unmet needs of racialized students and
        address racism on campus.
        \bifrt\ Council approves the creation of a service as the Service
        Creation and Cost-Benefit Proposal contingent on:

        \begin{itemize}
            \item CLAC oversight through regular reporting of the VP Student
                Life (formerly VP Internal), Services Manager, and relevant
                coordinators as necessary, and 

            \item Operational oversight of the implementation of the
                student-run service by CLAC including, but not limited to, the
                development of metrics of success and an outline of the
                deliverables the service shall be assigned, and

            \item The exploration and negotiation of future resource sharing
                arrangements currently provided by the Federation of Students
                with the relevant University bodies including, but not limited
                to the UW Office of Human Rights, Equity and Inclusion;
        \end{itemize}
    \end{motion}

    \begin{motion}
        \birt\ Council adds the word ``option'', such that the service 
        advocates for ``advocates optional anti-racism, anti-opression, ...''
        \movers{\alexander}{\seneca}

        \carries\ as friendly.
    \end{motion}

    \begin{motion}
        \birt\ Council removes ``For example, Ryerson Students’ Union funds and operates a student-run
        service called ``Racialised Students’ Collective'' that works with Ryerson’s
        community to eliminate racism and xenophobia both on and off campus'' through
        education and advocacy initiatives"
        \movers{\alexander}{\seneca}

        The motion fails.
    \end{motion}

    \carries. \harsh\ noted against. \nickta, and \alexander\ noted for.

\end{motion}
    
\heading{Feds Mascto}
\begin{motion}
    \whereas\ mascots are the embodiment of the collective identity and
    brand of an institution, and

    \whereas\ there is significant student desire for our official mascot
    and branding to incorporate a goose, and

    \whereas\ the Waterloo community at large has come to view geese as an
    unofficial symbol of the University of Waterloo, including through
    snapchat filtering and the sale of plush goose stuffed animals, now
    therefore,

    \birt\ the mascot of the Federation of Students shall be a Goose; and

    \bifrt\ the Council shall create an ad hoc “WaterFowl Mascot
    Committee”, to be chaired by the President or a representative
    thereof, which shall during a one (1) year period determine naming,
    design, and style of the Feds Mascot through student submission; and

    \bifrt\ this committee shall hold a survey of students for the naming
    of said goose, from which Council shall choose a name of the top
    contenders; and

    \bifrt\ Council tasks the Executive to work with the University to make
    a goose the official university mascot. 

    \bifrt\ The mascot of the Federation of Students shall not be contrued as
    official branding or marketing of the Federation.

    \movers{\seneca}{\jennifer}

    Some councillors expressed support for recognizing the goose as a signifier
    of the community.

    Motion to make this just or the current term. ruled out of order
    
    \carries. 
\end{motion}

\section*{New Business}

\heading{Cancel April Meeting}
\begin{motion}
    \birt\ Council cancels its next meeting.
\end{motion}

\heading{Partisan Activism in Services}
\begin{motion}
    \birt\ Council tasks \ppc\ to develop procedural restrictions on partisan
    activism of Federation of Student's services.

    \movers{\seneca}{\katie}
    \carries\ unanimously.

\end{motion}

\heading{Operational Oversight of Services}
\begin{motion}
    \birt\ Operational oversight of the implementation of the student-run
    services by CLAC including, but not limited to, the development of metrics
    of success and an outline of the deliverables the services shall be
    undertaken and PPC shall develop procedures that reflect this,

    \movers{\seneca}{\katie}
    \carries\ unanimously.
\end{motion}
