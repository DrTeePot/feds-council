\section*{Consent}

\heading{Approval of the minutes}
\begin{motion}
    \birt Council approve the minutes from July 16, 2017.
    \movers{\tristan}{\elizabeth}

    \seneca\ noted that he was recorded as absent, but was present.

    \carries unanimously.
\end{motion}

\heading{Councillor Reports}
\begin{information}
    No reports were given.
\end{information}

\section*{Executive Reports}

Action plans are still being developed, and will be presented at the 
next meeting. 

\heading{Report of the \pres}

    The \pres\ reported on the completion of a successful Orientation Week;
    they spoke at several events, and were requested to be present at many 
    faculty events.

    The Mental Health Commitee has posted the panelists who will be consulted 
    for the 5 main panels. The \pres\ attended a launch event that introduced 
    everyone; the panels will begin investigating mental health on campus in 
    the near future. A Q\&A panel will be available to take student questions
    after the General Meeting. 

    Working with Andrew and the other U15 student associations to develop 
    briefs on the student loan program, and aid for indigenous students. 
    They also will be advocating for increased access to undergraduate 
    research funding and for undergraduate work placements to count towards 
    international student residency. These priorities will be advocated 
    between November 6th to the 10th in Ottawa. The \vpe\ is also working on a 
    brief to better define the coalition and explain why the coalition is 
    unique to decision makers. 

    The Policy and Procedure Committee has been meeting bi-weekly. Overall 
    looking at expired and soon-to-be expired policies and determining how to 
    consolidate and reduce the number of policies. 
    
    A councillor asked if the Fall General Meeting had been scheduled. The 
    meeting will be held on October 25, 2017 at noon, first notice will be sent
    to students on September 25th, 2017. 

\heading{Report of the \vpi}

\begin{information}

    The \vpi\ organized three major events.  
    \begin{itemize}
        \item They spoke at events during Orientation Week, working with 
            various stakeholders and Orientation leaders to ensure the success
            of the week.
        \item Shortly after, Welcome Week greeted students back to campus. 
            It was very successful, with high attendance.  
        \item Finally, preperations are underway to ensure Wrap-up Week is as
            successful as the begining-of-term events. 
    \end{itemize}

    In their work towards increasing campus wellness, the \vpi\ is working 
    actively with Campus Wellness and uWaterloo Athletics to create a 
    coalition and unified front surrounding the welllness initiatives on campus. 
    Wrap up week will likely change to be focussed as a wellness week, and Feds
    and the other campus partners are aiming to be more preventative in their
    approaches to campus wellness. 

    The societies commisioner is still being hired, but they \vpi\ will be 
    scheduling the first Committee of President's meeting soon. Will be 
    discussing the events and issues that each society is focussed on and how 
    they can collaboratively run events and support students.

    Volunteer appreciation is also being redone, and the \vpi\ is beginning 
    by investigating how other unions appreciate volunteers and how we can do 
    better, there will be a presentation to the Campus Life Advisory Committee 
    with some preliminary ideas.

    The Gender and Sexual Diversity Workgroup has been meeting, beginning with
    education; later there will be 16 days of activism against gender violence,
    and more efforts to increase the visibility of this initiative and the 
    issues faced on this campus. 

    The Clubs Assistant that was hired at the beginning of the fall is now 
    actively supporting clubs and working diligently. 
    
\end{information}

\heading{Report of the \vpof}
\begin{information}

    The \vpof\ submitted a detailed report, and encourages those interested to
    read it and reach out with any questions. Some highlights include:
    \begin{itemize}
        \item A new Bomber menu has launched, with more food options and more 
            options for restrictive diets; 
        \item An update to International News' express menu is set to launch 
            in the winter, to allow students to get in and out quickly; 
        \item Fazing our straws at Bomber to eliminate plastic use; 
        \item More grab and go items (including possibly fried chicken and 
            potatoe wedges) will be available at International News, as the 
            existing options are very successful;
        \item Feds Used Books is very busy, there is a strong push for 
            afordable textbooks on campus;
        \item Revisting efficiencies on entry and exit to the Student Life
            Centre to address the traffic around campus bubble from 
            construction;
        \item Executive action plans have been posted online; 
        \item The new website will launch soon, it will be much better 
            organized and accessible on multiple devices;
        \item The budget was emailed to councillors and all concerns were 
            resolved prior to Board approval, it will be posted when the new 
            website is released;
        \item The opt-out period for the health and dental plans, is September
            5, 2017 till October 7, 2017; and,
        \item A legal insurance survey will be launched in October to re-visit
            whether students want legal coverage.            
    \end{itemize} 

    The \vpof\ also shared some updates on PharamPlus:
    \begin{itemize}
        \item PharmaPlus has released a list of prescription drugs that will be 
            covered and is set to launch in January;
        \item the \vpe\ and \vpof\ went to an OHIP meeting to determine more 
            informaiton on the PharmaPlus initiative; 
        \item Feds is creating documentation to inform students on the changes; 
        \item The are monitoring the current health and dental plan and 
            investigating potential changes for the future, however there
            will be no changes in the short term since PharmaPlus will launch 
            in the middle of a term year.
    \end{itemize}

    A councillor inquired why Council was not asked to approve the budget, and 
    was informed of the motion at the last meeting when Council delegated
    approval to the Budget Committee, provided that the budget was emailed to 
    Council for comments.  

    The \vpof\ is also re-working how the budget is constructed and presented to
    make it more accessible to students and make the budget process more 
    streamlined. 

    A councillor also asked about executive salaries, and why they were
    so high compared to other student unions. Salaries are set by board, so the 
    executive are not privy to those discussions due to conflict of interest.
    Councillors are encouraged to reach out to the chair of the Board of 
    Directors to find out more on this process. 

    A councillor asked what prompted the menu changes in commercial services,
    the \vpof\ responded that they are always looking to improve. 
    A major concern was that students couldn't get the food they wanted in the 
    Bombershelter Pub, and Feds wants to serve a variety of students. It is
    important to ensure that people have food to eat, so making sure that there
    are adequate vegetarian and vegan options is important to Feds. 
    The changes were also meant to address continual changes in the restaurant
    industry, and optimizing the menu based on which items were unpopular. 

    The councillor was concerned due to the food selection at the Cambridge 
    campus. The \vpof\ will touch base with the Campbridge councillor and 
    University administrators to find out more about how this problem can 
    be resolved. 

    A councillor inquired about what happens if Council doesn't approve the 
    budget, and if it would be similar to a government shut-down. An unapproved
    Council budget would result in no funding for clubs and services, but 
    salaries and commercial departments would still have funding since those
    are corporate areas with legal requirements. 

    A councillor asked about halal options at Feds commercial services.
    The \vpof\ responded that Grab-and-Go chicken was halal, and they were
    adding more halal options to the Bomber menu. 

\end{information}

\heading{Report of the \vpe}
\begin{information}

    The \vpe\ was not present. 


\end{information}

\section*{General Orders}

\heading{Hours of teaching buildings}
\begin{motion}
    \birt Council renews Policy 2. Hours of Teaching Buildings.
    \movers{\brian}{\seneca}

    \tristan gave a brief overview of the policy.

    \carries unanimously.
\end{motion}

\heading{Approval of Service Coordinators}
\begin{motion}
    \birt Council approves the service coordinators for Fall 2017 as presented. 
    \movers{\brian}{\ben}

    These are for one term. 

    \carries unanimously. 

\end{motion}

\heading{Library Renovations}
\begin{information}

    The Library gave a brief update on the ongoing renovations, see the
    attached slide-show. There were no comments.

\end{information}

\heading{Fall Reading Break}
\begin{information}
    \seneca\ gave a brief feedback on the independent survey he held. 

    Science constituents, the Science Society, Environment, Engineering, and 
    Arts also participated. 

    When asked what uWaterloo should do to compensate for adding days to the 
    Fall reading break, \begin{itemize} 
        \item 43\% said class and orientation should start early, 
        \item 38\% said that exams should be allowed on sundays, 
        \item some student did not want to extend reading week, and some wanted to
            get rid of it, and
        \item the least popular option was to shorten the time between the 
            end of classes and the start of exams, with weekend classes also 
            being very unpopular.
    \end{itemize} 

    In terms of what students thought they should be expected to do, 
    \begin{itemize}
        \item most students wanted to study and prepare for midterms,
        \item many students wanted reading week to be for mental health or
            relaxation, 
        \item a limited amount of students said that they should be able to do
            interviews during reading week,
        \item a little under half responded that students should check email 
            daily,
        \item very few students thought that exams or quizes should happen 
            during a reading break,
    \end{itemize}

    The vast majority of students supported the Fall reading break in its 
    current form, or as an extended break.  Many international students wanted
    a full week to go home, but an important consideration for these students
    is that sacrificing days at the start of the term results in no gain for 
    them. 

    Almost 550 student made comments, which will be sent to the council 
    mailing list. 

    A councillor voiced support for the data and work, and the policy that
    was created (see attached). 

    A councillor agreed with the plight of international students, and added 
    that it was impossible to move orientation sooner, as students arrive to
    Waterloo and immediately move into orientation week as is.  The councillor
    also inquired if demographic data was recorded in the survey. 

    \seneca\ responded that there was no demographic data gathered with the
    survey. This was intended as an preliminary poll to get some 
    data on the issue, and the amount of data gathered ended up being
    slightly overwhelming as-is. 

    The \vpof\ reassured councillors that the executive had disscussed some
    of the points of the working group and had come to the same conclusions
    as the data that was presented. The \vpof\ expressed gratitude for the
    data the was collected, as it will be useful, and encouraged
    councillors to reach out to the \vpe\ with any questions. 

    An environment councillor expressed that they had also collected data, and
    would like to add it to the data pool. 

    A councillor asked about the timeline for this policy and how much 
    time there was to gather data and propose policies; since University 
    calendars are set pretty far in advance there is not much time. 

    A councillor expressed that Feds should make another poll to gather more
    official data. Another councillor countered that this was a representative
    body that could make decisions in the best interest of students with
    student input. Another councillor asked that councillors talk to their
    student socities and gather feedback from them, and potentially send
    out emails about it. 

\end{information}

\heading{Resolution on the Extension of Fall Reading Break} 
\begin{motion}

    \birt Council perform a first reading of the presented 
    Resolution on the Extension of the Fall Reading Week, and
    \bifrt Councillors will communicate with constituents and societies to
    gather more feedback on the proposed resolution, and
    \bifrt the \vpe\ will draft a survey to distribute to constituents, and
    \bifrt the Policy and Procedures Committee will work with stakeholders to 
    create a more formal standing policy on mid term breaks. 
    \movers{\seneca}{\rebecca} 

    \carries unanimously. 

\end{motion}

\heading{Renewal of Policies}
\begin{motion}
    \birt Council renews policies 1, 3, 16, and 31 as presented by  
    the Policy and Procedures Committee. 
    \movers{\seneca}{\brian}

    \seneca\ gave a brief overview of the policies in question. They are
    fairly unmodified, with the exception of the policy on student study
    space where it is recommended that the percent of space allocated as 
    study space be increased to the standard determined by post-secondary
    research into study space on university campuses. 

    \carries unanimously

\end{motion}

\heading{Plastic Water Bottle Policy}
\begin{information}

    Based on a policy that was made at a recent Engineering Society Council 
    meeting, a councillor would like to create a more general policy at the 
    Feds level on how commercial services distribute bottled water, in the 
    effort to make Feds and the uWaterloo campus more environmentally
    sustainable. 

    A councillor voiced opposition to the policy due to the lack of water
    fountains in older buildings, a consideration for students 
    ingesting excessive amount of alcohol, and a view that the Federation
    shouldn't dictate the drinking choices of students. The councillor instead
    would ask that Feds advertise reusable options to students, but not 
    stop the sale of bottled water. 

    The \vpof\ expressed support for the removal of plastic water bottles from
    campus, citing the successes had on McMaster's campus with this initative. 
    However, the \vpof\ expressed concern that there were very few reusable
    bottle filling stations on campus, and that maybe Feds should take a
    leading stance on this but it was a concern that it would be difficult for
    students to refill the bottles. The \vpof\ would like to investigate this
    further as a campus wide stance on bottled water. The \vpof\ also brought 
    forward a potential middle ground of boxed water, which is much more
    efficient. Finally, he expressed that there should be a phase out period,
    as otherwise bottles will still end up in landfills. 

    A councillor brought up that other liquids in bottles would likely see
    increased sales, and so proposes that perhaps we should take a stance
    on all bottled liquids.  

    \jason\ expressed interest in meeting with the \vpof\ about this policy, 
    as well adding additional clauses to the policy regarding water-filling
    stations on campus. He also expressed disagreement that Feds removing water
    bottles does not infringe on student choice rights.

    The \pres\ added that this policy is very internal focussed, so may not 
    pass the Council procedure definition of policy, so this should be kept in 
    mind throughout policy development. 

    A councillor suggested keeping bottled water sales in buildings that did
    not have water-filling stations. Another suggestion was that Feds re-usable
    water bottles present a great branding opportunity. 

    The councillor that was earlier opposed suggested that this was the was
    the wrong solution to the right problem, re-iterating that students
    would likely be upset at Feds for discontinuing the sale of water 
    bottles and re-iterating that advertising is a more individualized 
    solution. 

    A councillor inquired about precedent for this kind of policy. The \vpof\ 
    responded that some items had been banned by the provincial government, but
    that it was rare for Council to bring forward explicit bans against 
    certain items. They also mentioned the existing policy on environmental
    sustainability that Councillors should read for context.

    A councillor brought up that it may be better to modify the environmental
    sustainability policy, and that the Policy and Procedures Committee could 
    work to create a more comprehensive policy proposal. 

    Another councillor expressed that a large number of students would be happy
    with this policy, however expressed that they were reprenting environment
    students. They also mentioned there was likely only a small number of
    students that would be unhappy with Feds for discontinuing bottled water. 

    A councillor added that if we are modifying the environmental 
    sustainability policy we should add clauses for recycling, as there are 
    many buildings that have no recycling facilities. 

    The \vpof\ will move water bottles out of the fore-front and will bring in
    boxed water instead. He will also investigate bringing in cheap but quality
    reusable water bottles. 

    \jason\ and \brian\ will connect to discuss this, and bring a more formal
    policy to the Policy and Procedures Committee.

\end{information}

\heading{Revoke Powers of Policy and Procedures Committee}
\begin{motion}

    \birt revokes the authority of the Policy and Procedure Committee to ratify
    policy or procedures without the approval of Council.
    \movers{\seneca}{\brian}

    \tristan\ proposed an ammendment to make this motion specific to the
    policies that were previously delegated to the committee, as the committee
    never had approval authority over all policy and procedure. This was done
    to clarify the exact effect of the motion.
    
    This was friendly with the mover and seconder. The policy now reads:

    \birt Council revokes the authority of the Policy and Procedures Committee
    to ratify ammendments to the \begin{itemize}
        \item Ancillary Fees,
        \item Preventing Discrimination, and
        \item Strong Representation
    \end{itemize} policies without the approval of Council. 

    \carries unanimously. 

\end{motion}

\heading{Freedom of Speech and Expression on Campus}
\begin{motion}
    \birt Feds adopt the distributed policy on Freedom of Speech as presented. 
    \movers{\alex}{\brian}

    The mover expressed that this policy is meant to treat all clubs the
    same regardless of their political belief, to allow campus police to
    intervene if protests happen, and to prevent protests from happening
    that would shut down events.

    Councillors expressed that polls show students are interested in seeing
    this policy be explicit on what speech is allowed, and there was concern
    optics of passing the policy in its current form. There was general support
    for a policy around free speech, and an interest in seeing a revised 
    policy. 

    The \pres\ expressed that the Policy and Procedures Committee was currently
    working on a policy on free speech based on one that had just expired, and
    invited \alex\ to come to a future meeting to discuss his thoughts further. 

    There was further concern for protections to the rights for campus
    community members to be on campus, particularly those who belong to
    marginalized communities.

    The mover made the argument that freedom of speech allows our community 
    to address and confront those who believe hateful things about other 
    people. He is also open to working with the Policy and Procedures 
    Committee in the future. 

    The \pres\ expressed support for the idea of this, but that the policy 
    should be rolled into our current policy on freedom of expression. 

    Another councillor expressed support for this policy to exist, but that
    what constitutes as free speech does needs to be defined. 

    The mover responded asking who decides what constitutes as hate speech, 
    and that students should stand up and address when they are uncomfortable. 

    A councillor expressed that there were enforced reasonable limits on
    freedom of expression, and that the law had precedent to any policies
    developed by Council. 

    A councillor would like to see an investigation into the safeguards that
    will be put into place for marginalized groups on campus. 

    \tristan\ proposed an ammendment to make this a first reading of the motion. 
    The ammendment is seen as friendly, and the motion now reads:

    \birt Council conduct a first reading of the Freedom of Speech policy
    as presented, and will recieve a revised version at the next meeting. 

    \seneca\ proposed referring the motion to the Policy and Procedures
    Committee for further development. 

    \birt Council refer this to the Policy and Procedures Committee to develop
    in concert with the current Freedom of Expression policy and the mover. 
    \movers{\seneca}{\brian}

    \carries unanimously.
\end{motion}

\heading{Council Meeting Dates}
\begin{motion}
    \birt Council set the dates for the Fall Council meetings
    \movers{\antonio}{\brian}

    \carries unanimously
\end{motion}


\section*{New Business}

\heading{Environment Councillor Report}
\begin{information}

    The councillors representing the faculty of environment submitted a report 
    during the meeting. See attached. 

    There is concern that students making expenses for student societies are
    not being repayed on time, and the possibility of societies having other
    bank accounts. 

    It was noted that some societies do have external bank accounts, but 
    most expenses are still processed through the societies accountant. 

    The \vpof\ expressed that the societies accountant now had a credit card to
    make society purchases easier. 

    A councillor also inquired if there was any investigations into online
    reimbursement forms. 

    The \vpof\ responded that it was a project going on at the moment, and that
    he could bring an update at the next meeting. 

    Councillors also expressed a desire to see more outreach from the Student
    Life Endowment Fund, as the fund seems under-utilized by societies. 

\end{information}

\heading{Open Seats on the Internal Administration Committee} 
\begin{information}

    A councillor inquired about open seats on IAC, the \vpi\ responded that
    there were none. 

\end{information}

\heading{Open Seats on Committees of Council}
\begin{information}

    A councillor inquired about open seats due to the term changing. 
    They were informed that Councillors and at large students will continue in
    their seats unless replaced. 

\end{information}

\begin{information}

    The architecture representative on Council raised the issue that 
    students on their campus feel neglected by Feds, and see a lot of 
    attention during the election but would like to see more attention 
    during the year. 

    The councillor would like to see executive office hours on their campus for
    a day at least once a month, ideally more frequently. 

    They would also like more support from the Feds and the University, as
    students are feeling a lot of pressure and many are working till 3am every
    night. 

    The excutives responded that they would be looking into that, and to
    reach out with any other concerns or suggestions.

\end{information}

