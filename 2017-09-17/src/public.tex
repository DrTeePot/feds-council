\section*{Consent}

\heading{Approval of the minutes}
\begin{motion}
    \birt Board approve the minutes from May 28, 2017.
    \movers{\tristan}{\elizabeth}

    \seneca\ noted that he was recorded as absent, but was present.

    \carries unanimously.
\end{motion}

\heading{Councillor Reports}
\begin{information}
    No reports were given.
\end{information}

\section*{Executive Reports}

Action plans are still being developed, and will be presented at the 
next meeting. 

\heading{Report of the \pres}

    Just finished oweek, spoke at several events throughout the week. Several
    faculties also requested the president's pressence at various academic events

    Mental health commitee has posted the panelists, 5 main panels. There
    was a launch event to introduce everyone and the panels will bee beginign
    investigating mental health on campus. A QandA panel will be created, 
    going to put it after the General meeting. 

    Working with andrew and the other u15 student associations to develop briefs
    on the student loan program, and aid for indigenous students. Also increased
    access to undergraduate research funding and counting work done during 
    undergrad as work for international student residency. Will be advocated nov
    6-10 in ottawa. Aso working on a brief to define the coalition better to
    make it stand out and why the coalition is unique. 

    PPC has been meeting regularly, bi-weekly. Overall looking at expired and
    soon to be expired policy and looking at how to consolidate and reduce
    the number of policies. 
    
    A councillor asked if the fall GM had been scheduled. October 25 at noon
    , notice will be september 25th. 

\heading{Report of the \vpi}

\begin{information}

    Events: 
    - spoke at events during orientation, worked with various stakeholders and
    o week leaders 
    - welcome week was very successful, high attendance. 
    - ramping up for wrap up week

    Wellness:
    - working actively with campus wellness and athletics to create a 
    coalition and unified front with the welllness initiatives on campus
    to ensure everyone is on the same page. Wrap up week will likekly change
    to be focussed as a wellness week. bein more preventative

    societies 
    - still hiring, having the first cops meeting soon. Will be talking about
    whats happening in each society and how they can come together to 
    run events and support students

    Volunteer appreciation
    - went to an event to investigate how other unions appreciate volunteers
    and how we can do better, will be presenting to CLAC with some ideas

    Gender and sexual diversity workgroup
    - education; 16 days of activism against gender violence, visibility 

    Clubs assistance that was hired in the beginning of fall is now actively
    supporting clubs and working diligently. 
    
\end{information}

\heading{Report of the \vpof}
\begin{information}

    Lots of detail in his report, please read. 
    
    Bomber menu has launched, more food options and more accomodations for
    different diets. 

    Express menu to launch in winter, to allow students to get in and out 
    quicker. 

    Fazing our straws at Bomber to eliminate plastic use. 

    INews is creating more grab and go items, they are very successful. Fried
    chicken and potatoe wedges are some of the options

    Feds used books is super busy, strong push for afordable textbooks on 
    campus

    Revisting efficiencies on entry and exit to the building to address the 
    cluster around campus bubble due to construction.

    Exec action plans are all posted online, go check them out. 

    Pharmaplus has come out with a list of prescription drugs they will be 
    covering, andrew and brian went to an OHIP meeting to determine more 
    informaiton.  set to launch in january. Creating documentation to 
    inform students. Monitoring the health and dental plan and looking 
    at what might be needed for the future, but not making any short term
    decisions (pharmaplus will launch in the middle of a term year). 

    budget has been approved, will be posted. IT was emailed to councillors
    and recieved some questions but nothing unresolved. 
    
    New website will launch soon, will be much better organized and accessible
    on multiple devices. 

    Opt-out health and dental, september 5th till october 7th. 

    legal insurance survey. A survey has been drafted with student care and
    will be launched in october to re-visit whether students want legal 
    coverage.

    a councillor asked why council didn't have to approve the budget. The 
    previous meeting, council delegated it's approval to the budget committee,
    provided that the budget was sent to council for comments.  

    brian is also re-working how the budget is constructed and presented to
    make it more accessible to students and make the budget process more 
    streamlined. 

    A councillor also asked about our executive salaries, and why they were
    so high compared to other unions. Salaries are set by board, so the 
    executive are not privy to those situations due to conflict of interest. 

    a councillor asked what prompted the menu changes in commercial services,
    the vpof responded that they are always looking to improve and ensuring 
    they aren'tn in a rut. A major concern was that students couldn't get
    the food they wanted in the bomber, and we want to serve a variety of 
    students. It is also importnat to ensure that people have food to eat, 
    so making sure that there were adequate vegetarian and vegan menus. 
    Also addressing continual changes in teh restaurant industry, and 
    looked at how things were selling. 

    The councillor was concerned due to teh selection at the cambridge campus.
    the vpof will touch base with teh campbridge councillor and admins to
    ask. 

    A councillor inquired what ahppens if council doesn't approve the budget,
    and if it would be like a government shut down. Clubs and services would 
    not have funding, but salaries and commercial items would still have 
    funding. 
    
    A councillor asked about hallal options at commercial services, and the vpof
    responded that Grab and go chicken was hallal and more options would be
    moving into the bomber. 

\end{information}

\heading{Report of the \vpe}
\begin{information}

    Absent. 


\end{information}

% \section*{Business Arising from the Minutes}

% \heading{Ratification of at-large members to committees}

% \begin{motion}
%     \birt Council ratifies Vaishnavy Gupta as an at-large member of the 
%     Presidents Advisory Committee, and 
%     \bifrt Council ratifies Hannah Sesink as an at- large member of the 
%     Internal Administration Committee.
%     \movers{\brian}{\tomson}

%     \carries unanimously. 
% \end{motion}

\section*{General Orders}

\heading{Hours of teaching buildings}
\begin{motion}
    \birt
    Brief description from Tristan

    \brian \seneca

    \carries unanimously
\end{motion}

\heading{Approval of Service Coordinators}
\begin{motion}
    \birt Council approves the service coordinators as presented. 
    \moved{\brian}{\easton}

    These are for one term. 

    \carries unanimously. 

\end{information}

\heading{library}
\begin{information}
    
    Library updat eon construction. No comments

\end{information}

\heading{Fall Reading Break}
\begin{information}
    \seneca gave a brief feedback on the survey held. 

    Science constituents, science society, env and engineering and arts also
    participated. 

    What should uw do to compensate for added days of reading week. 
    43\% said class and orientaiton should start early. 

    Allow exams (38\%) on sundays

    least popular was to shorten the time between the end of classes and the
    start of exams. 

    Some student did not want to extend reading week, and some wanted to
    get rid of it. 

    One student suggested condensing two days of classes into one day. 

    Weekend classes were also very unpopular. 

    second question was what students should be expected to do. 

    Most students wanted to study and prepare for midterms. 

    Many students also said that reading week should be for mental health or
    relaxation. 

    A limited amount of students said that they should be able to do interviews
    during reading week. 

    A little under half said that students should check email daily

    And very few students wanted exams or quizes. 

    Almost 550 student comments, this will be sent to council. 

    The vast majority of students supported extensions or keeping it the same. 

    Many international students wanted a full week to go home, but an 
    important consideration si that less summer is net zero. 

    A councillor voiced support for the data and work, and the policy that
    was created. 

    A councillor agreed with the plite of internation students, and added that
    it was impossible to move orientation sooner, as students come in and 
    must immediately move into orientation week as is.  The councillor
    also inquired if the year of the student was recorded in the survey. 

    \seneca\ responded that there was no demographic data gathered with the
    survey. This was intended as an original poll to get some preliminary 
    data on the issue, and the amount of data gathered ended up being
    slightly overwhelming. 

    The \vpof\ reassured councillors that the executive had disscussed some
    of the points of the working group and had come to the same conclusions
    as the data that was presented. The \vpof\ expressed gratitude for the
    data the was collected, as it will be useful, and encouraged
    councillors to reach out to the \vpe\ with any questions. 

    An environment councillor expressed that they had also collected data, and
    would like to add it to the data pool. 

    A councillor asked about the timeline for this policy and how much 
    there was. 

    A councillor expressed that we should make another poll to gather more
    official data. Another councillor countered that thsi was a representative
    body that could make decisions in teh best interest of students with
    student input. Another councillor asked that councillors talk to their
    student socities and gather feedback from them, and potentially send
    out emails about it. 

\end{information}

\heading{Resolution on the Extension of Fall Reading Break} 
\begin{motion}

    \birt Council perform a first reading of the presented 
    Resolution on the Extension of the Fall Reading Week, and
    \bifrt Councillors will communicate with constituents and societies to
    gather more feedback on the proposed resolution, and
    \bifrt the \vpe\ will draft a survey to distribute to constituents, and
    \bifrt the Policy and Procedures Committee will work with stakeholders to 
    create a more formal standing policy on mid term breaks. 
    \mover{\seneca}{\rebecca} 

    \caries unanimously. 

\end{motion}

\heading{Renewal of Policies}
\begin{motion}
    \birt Council renews policies 1, 3, 16, and 31 as presented by  
    the Policy and Procedures Committee. 
    \movers{\seneca}{\brian}

    \seneca\ gave a brief overview of the policies in question. They are
    fairly unchanged, with the exception of the policy on student study
    space where it si recommended that teh percent of space allocated as 
    study space be increased. 

    \caries unanimously

\end{motion}

\heading{Plastic Water Bottle Policy}
\begin{information}

    Based on a policy that was made at EngSoc, a councillor would like to create
    a more general policy at the Feds level on how commercial services 
    distribute bottled water, in the effort to make this more environmentally
    sustainable. 

    A councillor voiced opposition to the policy due to lack of water
    fountains in older buildings, a consideration for students 
    ingesting accessive amount of alcohol, and that the Federation shouldn't
    dictate the drinking choices of students. The councillor instead
    would ask that Feds advertise reusable options to students, but not 
    stop the sale of bottled water. 

    The \vpof\ expressed support for plastic water bottles to be removed from
    campus, citing the successes had on McMasters campus with this initative. 
    However, the \vpof\ expressed concern that there were no reusable bottle 
    filling stations on campus, and that maybe Feds should take a leading stance
    on this but it was a concern that it would be difficult for students
    to refill the bottles. The \vpof\ would like to investigate this further
    as a campus wide stance on bottled water. The \vpof\ brought 
    forward a potential middle ground of boxed water which is much more
    efficient. Finally, he would like a phase out period, as otherwise 
    bottles will still end up in landfills. 

    A councillor brought up that other liquids in bottles would likely see
    increased sales, and so proposes that perhaps we should take a stance
    on all bottled liquids.  

    \jason\ expressed interest in meeting with the \vpof\ about this policy, 
    as well adding lines regarding water-filling stations on campus. He also
    expressed disagreement that Feds removing water bottles does not infringe
    on student choice rights.

    The \pres\ added that this policy is a bit internal, so may not pass
    the Council definition of policy, so this should be kept in mind
    throughout policy development. 

    A councillor suggested keeping bottled water sales in buildings that did
    not have water-filling stations. Another suggestion is that Feds re-usable
    water bottles present a great branding opportunity. 

    The councillor that was earlier opposed suggested that this was the was
    the wrong solution to the right problem, re-iterating that students
    would likely be upset at Feds for discontinuing the sale of water 
    bottles and re-iterating that advertising is a more individual solution. 

    A councillor inquired about precedent for this kind of policy. The \vpof\ 
    responded that some items had been banned by the provincial government, but
    that it was rare for Council to bring forward explicit bans against 
    certain items. They also mentioned the existing policy on environmental
    sustainability that Councillors should read for context.

    A councillor brought up that it may be better to modify the environmental
    sustainability policy, and that PPC could work to create a more 
    comprehensive policy. 

    Another councillor expressed that a large number of students would be happy
    with this policy, however expressed that they were reprenting environment
    students. They also mentioned there was a small number of students that
    would be unhappy with Feds discontinuing water bottles. 

    A councillor added that if we are modifying the environmental 
    sustainability policy we should add clauses for recycling, as there are 
    many buildings that have no recycling facilities. 

    The \vpof\ will move water bottles out of the fore-front and will bring in
    boxed water instead. He will also investigate bringing in cheap but quality
    reusable water bottles. 

    \jason\ and \brian\ will connect to discuss this, and bring a more formal
    policy to \ppc.

\end{information}

\begin{motion}

    \birt Council recoke the authority of Council ..
    \movers{\seneca}{\brian}


    \tristan proposed an ammendment to make this motion specific to the policies
    that were delegated previously. This was friendly with the mover and 
    seconder. The policy now reads:

    \birt Council revokes the revokes the authority of the Policy and Procedures
    Committee to ratify ammendments to the Ancillary Fees and Preventing Discrimination
    and Strong Representation, policies without 
    the approval of Council. 

    \carries unanimously. 

\end{motion}

\heading{Freedom of Speech and Expression on Campus}
\begin{motion}
    \birt Feds adopt the distributed policy on Freedom of Speech as presented. 
    \movers{\alex}{\brian}

    The mover expressed that this policy is meant to treat all clubs the
    same regardless of their political belief, to allow campus police to
    intervene if protests happen, and to prevent protests from happening
    that would shut down events. 

    A councillor expressed that polls show students are interested in seeing
    this policy be explicit on what speech is allowed, and how this policy
    will be seen. They expressed support for a policy around free speech, but
    is wary of it passing in its current form. 

    The \pres\ expressed that \ppc\ was working on a current policy on 
    free speech based on one that had just expired, and invited \alex\ to
    come to a future meeting. 

    Another councillor expressed concern for how this would appear if passed
    as is, and further concern for the protections around civility in argument
    and the protections being given to the rights for campus community members
    to be in the space, particularly those who belong in marginalized 
    communities. 

    The mover made the argument that freedom of speech allows our community 
    to address and confront those who believe hateful things about other 
    people. He is also open to working with \ppc\ in the future. 

    The \pres\ expressed support for the idea of this, but that the policy 
    should be rolled into our current policy and freedom of expression. 

    Another councillor expressed support for this policy to exist, but that
    what constitutes as free speech does need to be defined. 

    The mover responded asking who decides what constitutes as hate speech, 
    and that students should stand up and address when they are uncomfortable. 

    \tristan\ proposed an ammendment to make this a first reading of the motion. 
    it is seen as friendly, the motion now reads:

    \birt Council conduct a first reading of the Freedom of Speech policy
    as presneted, and will recieve a revised version at the next meeting. 

    A councillor expressed that there were enforced reasonable limits on the
    freedom of expression, and that the law had precedent. 

    \birt Council refer this to PPC to develop in concert with the current
    Freedom of Expression policy and the mover. 
    \movers{\seneca}{\brian}

    A councillor would like an investigation into the safeguards that will be
    put into place for marginalized groups. 

    \caries unanimously.  
\end{motion}

\heading{Council Meeting Dates}
\begin{motion}
    \birt Council set the dates for the Fall Council meetings

    \movers{\antonio}{\brian}

    \caries unanimously
\end{motion}


\section*{New Business}

\heading{Environment Councillor Report}
\begin{information}

    The environment councillors submitted a report during the meeting. See
    attached. 

    There is concern that students are not being repayed on time in student 
    societies, and the possibility of societies having other bank accounts. 

    The \vpof\ expressed that the society accountant now had a credit card to
    make society purchases easier. 

    A councillor also inquired if there was any investigations into online
    reimbursement forms. 

    The \vpof\ responded that it was a project going on at the moment, and that
    he could bring an update at the next meeting. 

    Wanting more outreach from SLEF. As there has been not a lot of visitility. 

\end{information}

\begin{information}

    A councillor inquired about open seats on IAC, the \vpi\ responded that
    there were none. 

\end{information}

\begin{Open seats}

    A councillor inquired about open seats due to the term changing. 
    Coucnillors and at large students will continue in their seats unless
    replaced. 

\end{Open seats}

\begin{information}

    Accredition happens in architecture. Students raised the issue that 
    studentas on that campus feel neglected by Feds, and see a lot of 
    attention during the election but would like to see more attention 
    during the year. 

    The councillor would like to see executive office hours, having the Feds
    executives on campus in the satelite campuses for a day at least
    once a month, ideally more frequently. 

    Would like more support, students are feeling a lot of pressure and are 
    working up till 3am every night. 

    The excutives responded that they would be looking into that, and to
    reach out with any other concerns or suggestions.

\end{information}

Adjourned at 3:07.

\movers{elizabeth}{brian}

\carries unanimously. \\
